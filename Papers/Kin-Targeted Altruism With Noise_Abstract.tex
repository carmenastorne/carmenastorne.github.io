\documentclass[12pt]{article}

\usepackage{hyperref}
\usepackage{amsmath}
\usepackage{amsthm}
\usepackage{amssymb}
\usepackage{setspace}
\usepackage{graphicx}
\usepackage{harvard}
\usepackage[toc,page]{appendix}

\addtolength{\hoffset}{-0.2cm}
\addtolength{\voffset}{-1.0cm}
\addtolength{\textwidth}{1cm}
\addtolength{\textheight}{1.6cm}

\newtheorem{theorem}{Theorem}
\newtheorem{acknowledgement}{Acknowledgement}
\newtheorem{algorithm}{Algorithm}
\newtheorem{axiom}{Axiom}
\newtheorem{case}{Case}
\newtheorem{claim}{Claim}
\newtheorem{conclusion}{Conclusion}
\newtheorem{condition}{Condition}
\newtheorem{conjecture}{Conjecture}
\newtheorem{corollary}{Corollary}
\newtheorem{criterion}{Criterion}
\newtheorem{definition}{Definition}
\newtheorem{example}{Example}
\newtheorem{exercise}{Exercise}
\newtheorem{lemma}{Lemma}
\newtheorem{notation}{Notation}
\newtheorem{problem}{Problem}
\newtheorem{proposition}{Proposition}
\newtheorem{remark}{Remark}
\newtheorem{solution}{Solution}
\newtheorem{summary}{Summary}

%%%%%%%%%%%%%%%%%%%%%%%%%%%%%%%%%%%%%%%%%%%%%%%%%%%%%%%%%%  TITLE  %%%%%%%%%%%%%%%%%%%%%%%%%%%%%%%%%%%%%%%%%%%%%%%%%%%%%%%%%%
\title{Kin-targeted Altruism With Noise\thanks{The author has benefited from insightful comments from David Levine and John Nachbar. I also thank all participants in the WUSTL Monday Book and Discussion Group, and the Missouri Economics Conference.}}
\author{Carmen Astorne-Figari \thanks{Department of Economics, University of Memphis, 427 Fogelman Admin Building, Memphis, Tennessee 38152;  e-mail address: \href{mailto:cmstrnfg@memphis.edu}{\tt cmstrnfg@memphis.edu}.}}

\date{This version: \today}

%%%%%%%%%%%%%%%%%%%%%%%%%%%%%%%%%%%%%%%%%%%%%%%%%%%%%%%%%%%%%%%%%%%%%%%%%%%%%%%%%%%%%%%%%%%%%%%%%%%%%%%%%%%%%%%%%%%%%%%%%%%%%
\begin{document}
\maketitle
%%%%%%%%%%%%%%%%%%%%%%%%%%%%%%%%%%%%%%%%%%%%%%%%%%%%%%%%  ABSTRACT  %%%%%%%%%%%%%%%%%%%%%%%%%%%%%%%%%%%%%%%%%%%%%%%%%%%%%%%%%
\begin{abstract}
\thispagestyle{empty}
\bigskip
Can pure altruism generate strategic altruism when kin recognition is noisy? This paper studies a prisoners' dilemma played between two individuals who exhibit altruistic preferences towards kin. The probability that a player's opponent is kin is common knowledge, but, instead of directly observing whether or not the other player is related, each player observes a noisy private signal. When the game is played once, players cooperate only with those identified as kin. However, when the prisoners' dilemma is played for two periods instead of one, uncertainty about relatedness brings strategic considerations into the game even if the odds of being related are small. There are Perfect Bayesian Equilibria in which players cooperate in the first round even after getting a negative kin signal. Since a player can make inferences about her opponent's signal based on first period actions, a non-relative mimics kin to induce cooperation in the second period.
\\
\bigskip \bigskip

\noindent \textit{Keywords}: Altruism; noisy signaling; prisoners' dilemma\\

\noindent \textbf{JEL Classification Numbers:} C72, C73, D64
\end{abstract}

\newpage
%%%%%%%%%%%%%%%%%%%%%%%%%%%%%%%%%%%%%%%%%%%%%%%%%%%%%%  INTRODUCTION  %%%%%%%%%%%%%%%%%%%%%%%%%%%%%%%%%%%%%%%%%%%%%%%%%%%%%%%
%\setcounter{page}{1} \doublespacing \section{Introduction}

%Can repetition increase cooperation when players' preferences exhibit kin-targeted altruism and kin recognition is noisy? As suggested by kin selection models (Hamilton, 1964), players are altruistic towards relatives when such behavior increases inclusive fitness. Inclusive fitness, also known as ``Hamilton's rule'', is defined as an average of an individual's own survival probability and that of her kin's, weighted by the degree of relatedness between them. Favoring only kin and not those who are not related requires kin recognition mechanisms, which rely on using noisy signals to discriminate between kin and non-kin. We say that kin-targeted altruism occurs when players condition their strategies on these signals, choosing to cooperate only with those who exhibit the signal associated to kin. If, after observing the signals, the prisoners' dilemma is played for two periods instead of one, given that the signals are private, players' strategies may be used to reveal or to conceal information about the signal that they observed. Even if the probability of being related to one's opponent is not very high, repetition may generate an incentive to cooperate, either to conceal or to reveal such information. 
%We ask whether repeating the game for one more period can enhance the initial level of cooperation from the one-period game. 

%Depending on the payoffs of the game and the degree of noise, we distinguish two cases. In both of them, a player who observes the relatedness signal acts like a behavioral type: either a grim trigger or a blind cooperator. In the first case, if the grim trigger punishment is strong enough, cooperation increases because players who observe the unrelatedness signal prefer to conceal their signal in order to avoid the grim trigger punishment (and defect against a cooperating opponent in the second period). However, if the grim trigger punishment is not strong enough, repetition generates the opposite result. In the second case, cooperation always increases, because knowing that at least one player observed the relatedness signal is sufficient for both players to want to cooperate.


\end{document}
