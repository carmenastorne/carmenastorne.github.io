\documentclass[12pt]{article}

\usepackage{amsmath}
\usepackage{amsthm}
\usepackage{amssymb}
\usepackage{setspace}
\usepackage{graphicx}
\usepackage{harvard}
\usepackage{color}
\usepackage{subcaption}
\usepackage{comment}
\usepackage{float}
\usepackage[toc,page]{appendix}
\usepackage{relsize}
\usepackage[margin=1.0in]{geometry}
\usepackage[colorlinks=true,linkcolor=red,citecolor=blue,filecolor=blue,urlcolor=blue]{hyperref}

\renewcommand{\appendixpagename}{\Large{Appendix}}
\renewcommand{\appendixtocname}{Appendix}


\newtheorem{theorem}{Theorem}
\newtheorem{acknowledgment}{Acknowledgment}
\newtheorem{algorithm}{Algorithm}
\newtheorem{axiom}{Axiom}
\newtheorem{case}{Case}
\newtheorem{claim}{Claim}
\newtheorem{conclusion}{Conclusion}
\newtheorem{condition}{Condition}
\newtheorem{conjecture}{Conjecture}
\newtheorem{corollary}{Corollary}
\newtheorem{criterion}{Criterion}
\newtheorem{definition}{Definition}
\newtheorem{example}{Example}
\newtheorem{exercise}{Exercise}
\newtheorem{lemma}{Lemma}
\newtheorem{notation}{Notation}
\newtheorem{problem}{Problem}
\newtheorem{proposition}{Proposition}
\newtheorem{remark}{Remark}
\newtheorem{solution}{Solution}
\newtheorem{summary}{Summary}

\begin{document}
\hyphenation{mar-gi-nal}

\title{Advertising for Consideration\footnote{The authors would like to thank David Ewoldsen, Andreas Hefti, David Levine, John Nachbar, Bruce Petersen and seminar participants at Binghamton University, the Federal Communications Commission, Michigan State University, the University of Memphis, and the Southern Economic Association Conference (2016). We are also grateful for the valuable research assistance of Stephen M. Lee. This work received grant and/or travel support from the Fogelman College of Business \& Economics at the University of Memphis and the Quello Center at Michigan State University. This research support does not imply endorsement of the research results by either the Fogelman College, the University of Memphis, the Quello Center, or Michigan State University. This work stems from an earlier presented working paper concerning brand advertising under limited attention.}}

\author{Carmen Astorne-Figari\thanks{Corresponding author. Department of Economics, University of Memphis, 427 Fogelman Admin Building, Memphis, Tennessee 38152;  e-mail address: \href{mailto:cmstrnfg@memphis.edu}{\tt cmstrnfg@memphis.edu}.} \and  Jos\'{e} Joaqu\'{i}n L\'{o}pez\thanks{Department of Economics, University of Memphis, 415 Fogelman Admin Building, Memphis, Tennessee 38152;  e-mail address: \href{mailto:jjlopez@memphis.edu}{\tt jjlopez@memphis.edu}.} \and Aleksandr Yankelevich\thanks{Federal Communications Commission, 445 12th Street, SW, Washington, DC 20554. e-mail address: \href{mailto:alyankelevich@yahoo.com}{\tt alyankelevich@yahoo.com}. Aleks Yankelevich is a Senior Economist at the Federal Communications Commission. This work was completed while he was employed by the Brandmeyer Center for Applied Economics at the University of Kansas and by the Quello Center at Michigan State University. The analysis and conclusions set forth are those of the authors and do not necessarily represent the views of the Federal Communications Commission, other Commission staff members, or the U.S. Government.}}

\date{This version: \today}


\maketitle 
%%%%%%%%%%%   ABSTRACT   %%%%%%%%%%%%%%%%%

\begin{abstract}

\thispagestyle{empty}

We analyze markets where firms competing on price advertise to increase the probability of entering consumers' consideration sets. We find that moderately costly advertising allows firms to raise prices and possibly profits by reducing the fraction of price-conscious consumers, and by segmenting the market according to whether or not consumers consider the lower priced firm. However, when the cost of advertising is sufficiently low, advertising leads to a prisoners' dilemma that adversely impacts profits without affecting expected prices. \\

\noindent \textit{Keywords}: Advertising, Bounded Rationality, Consideration Sets, Oligopoly, Price Dispersion\\

\noindent \textbf{JEL Classification Numbers:} D03, D21, D43, L13, M37

\end{abstract}

\newpage


%%%%%     INTRODUCTION    %%%%%%%

\setcounter{page}{1} \onehalfspacing 




\section{Introduction}\label{Introduction}

Experimental studies of consumer behavior suggest that, in many purchasing situations, consumers experience a lack of motivation to devote substantial cognitive processing efforts to brand selection. Instead, consumers' attention is typically restricted to a subset of all feasible alternatives, referred to in the economics and marketing literature as the consideration set.\footnote{See Roberts and Lattin (1997) for a survey of the marketing literature on consideration sets.} For example, Hoyer (1984) shows that, when purchasing detergents at the supermarket, most people look at only one brand. In this context, an important role of advertising is aimed at entering consumers' consideration sets.

We study the role of advertising in the formation of consideration sets, and its effects on equilibrium prices and profits in markets for homogeneous goods. In our paper, advertising refers only to actions that seek to build ``top-of-mind'' awareness in consumers. For example, firms may do this by running sponsored ads online, buying better shelf space at a supermarket, or including coupons in a circular. When successful, such advertising permits one firm to become ``prominent,'' and more likely to be considered by consumers. 

To undertake our analysis, we construct a model for consideration set formation that allows us to introduce the effect of advertising in a convenient way. In our duopoly framework, consumers perform two independent attempts to evoke\footnote{\label{evoke}To bring or recall to the conscious mind. See Oxford Learner's Dictionary. Available at https://en.oxforddictionaries.com/definition/evoke.} either of two firms selling a good that consumers want for inclusion in their consideration set. During each attempt, consumers evoke one of the firms in the market with a certain probability, and it is possible that the same firm is evoked both times.  Absent advertising, each firm has equal probability of being evoked on each attempt, making each firm equally likely to enter a consumer's consideration set. 
However, a firm may have the incentive to pay to ``advertise for consideration'' in attempt to become prominent. The extent to which the firm may wish to do so depends on the cost of advertising and the increase in the firm's probability of entering the consideration set of any particular consumer, i.e., the gain from prominence.  

This simple process of consideration set formation creates ex-post consumer heterogeneity: some consumers will consider one firm, while others will consider both. Similarly to models of sales (Varian 1980) and search (Burdett and Judd 1983, Stahl 1989) in markets for homogeneous goods, ex-post heterogeneity in the size of consideration sets across consumers results in price dispersion, as firms choose between running sales to attract price-conscious consumers who consider two firms (henceforth referred to as shoppers), and charging the monopoly price to exploit consumers who consider only one (referred to as captives).\footnote{Note that both dispersion in the size of different consumer consideration sets and consideration sets of size one are consistent with previous marketing studies (e.g., Hoyer 1984, Lapersonne et al. 1995), as are differences in the probability that different firms make it into the consideration set following advertising (Mitra 1995, Allenby and Ginter 1995).} 

The profitability of advertising, relative to a counterfactual scenario in which advertising is not permitted, turns out to be non-monotonic in the cost of advertising and the gain from becoming prominent. Perhaps not surprisingly, when either the cost of advertising is exceedingly low or the gain from prominence is very high, advertising is unprofitable because firms advertise too much, reducing the probability that either one will become prominent relative to the other. In contrast, when advertising is relatively expensive or the gain from prominence not very high, firms will engage in behavior reminiscent of co-opetition (\`{a} la Brandenburger and Nalebuff 1996), alternating between advertising and selling to a higher fraction of captives at a higher price, and serving a lower fraction of captives together with shoppers, but at a lower price. Put differently, advertising serves as a profit-enhancing coordination device, but only insofar as it is not so attractive that it induces an ad war among competitors. In this case, the probability of duplicating rivals' advertising efforts is relatively low, which could lead to higher expected profits. If, however, advertising is sufficiently expensive or ineffective, it again ceases to be profitable and may lead firms to refrain from advertising at all.

Our results concerning the price and profit-enhancing features of advertising for consideration are robust to various alternative formulations. In Section~\ref{Oligopoly}, we extend our model to a formulation with more than two firms. Moreover, as we discuss in Section~\ref{Conclusion} (and show in detail in the Supplemental Appendix), our main duopoly results persist whether firms choose advertising and pricing strategies simultaneously or in sequence as well as, to a degree, when firms have different costs of advertising.



%%%%%%   RELATED LITERATURE    %%%%%%%%
\subsection{Related Literature}\label{Related Literature} 

It has long been known that advertising expenditures may be excessive or insufficient. For instance, Marshall (1919, pp. 304-307) suggested that much expenditure on advertising is combative and consequentially, socially wasteful. As later studies suggested, firms may overprovide or underprovide advertising, whether advertising is persuasive (Kaldor 1950; Dixit and Norman 1978), complementary to consumer utility (Becker and Murphy 1993) or informative (Grossman and Shapiro 1984).\footnote{Bagwell (2007) expounds on the different forms of advertising in depth.}

In this paper, we offer a novel variation on advertising over- or under-provision by studying the pricing and profitability of advertising intended to influence the likelihood of consideration by a consumer. Such advertising neither alters nor augments utility of consuming a product, nor seeks to inform, as assumed by numerous studies of advertising for homogenous goods (Butters 1977; Robert and Stahl 1993; Stahl 1994, 2000; Baye and Morgan 2001). In contrast, as in more recent studies of consideration seeking, our study supposes that consumers might base a purchase decision on, say, the attractiveness or prominence of product placement.

The equilibrium in our model bares some resemblance to that in most studies of informative advertising for homogeneous goods, with, as suggested earlier, differences in the number of brands that consumers consider leading to price dispersion. However, there is a marked difference between the equilibrium outcome of interest in our model and those in many models of informative advertising for a homogenous good. First, in most informative advertising models, advertised prices are lower than unadvertised ones, whereas the opposite tends to hold true in our case.\footnote{As we discuss below, Janssen and Non (2008) is a notable exception.} Intuitively, this follows because when promotional expenditures make demand more elastic---as when more consumers are better informed---it is optimal to reduce prices on the product being promoted. When, as is the case in our model, promotional expenditures make demand less elastic by reducing the number of shoppers, the prices of promoted products should rise. Perhaps more starkly, in many models of informative advertising, unless advertising is necessary for the market to exist, firms would have been better off not advertising to begin with. As we show, while advertising for consideration can force firms into a prisoners' dilemma, there is a range of advertising costs such that advertising is more profitable than not.

Within the literature on informative advertising, the closest related work is that of Janssen and Non (2008). Janssen and Non study a homogenous good duopoly in which consumers are \textit{ex ante} heterogeneous: a fraction obtain price information at no cost and consider both goods; the remainder pays a search cost to learn each firm's price, and in equilibrium, will search no more than one. Advertisements that inform consumers of the existence and price of the product substitute search for consumers who must pay to do so and are sufficient to assure that these consumers consider the advertised product.\footnote{Indeed, without obtaining the information found in the ad, some such consumers may stay out of the market altogether} Consequently, as in our model, firms can use advertising to profitably segment the market. Despite this similarity, there are several important distinctions between our model and that of Janssen and Non. 

First, in Janssen and Non (2008), when one firm advertises while its rival does not---and using our terminology, becomes prominent---it successfully captures every consumer who would otherwise pay to search, charging them a higher price than its rival, who only sells to consumers who consider both goods \textit{ex ante}. In contrast, in our model, even if a firm attains prominence, advertisements are insufficient to guarantee consideration; some consumers are captive to (only consider) the non-advertising firm. We view this as more realistic, as some consumers who see ads ignore advertised products and possibly buy unadvertised ones at a lower price. Second, in line with other models of informative advertising, in Janssen and Non (2008), compared to a baseline without advertising, ads pull consumers in the opposite direction to that in our framework. Thus, in Janssen and Non (2008), an uptick in (the probability of) advertising unambiguously increases the expected fraction of consumers who consider two firms.\footnote{More precisely, a higher probability of advertising raises the probability that all consumers will consider two firms.} In contrast, ads in our model do not convey information, seeking instead to influence the probability that a product will be considered by any consumer. Thus, in our model, as firms advertise more, starting from a relatively low level, they can reduce the expected fraction of consumers who consider two firms, which could lead to expected price increases. 

Finally, that Janssen and Non treat advertising as completely informative implies that firms would never advertise with certainty to avoid homogenous Bertrand competition. In fact, because search and advertising are substitutes, there is no clear room for wasteful advertising in Janssen and Non (2008). A high advertising intensity means that consumers save on search costs and conversely, when consumers engage in substantial search effort, firms save on advertising expenditures. Therefore, there are social costs associated with either very low or very high advertising, but the distribution of those costs is what varies. In our model, as the cost of advertising falls, firms step up their competition for captive consumers, leading them to over-advertise by spending money for ads without increasing their probability of being considered. 

Our work is also closely related to a number of studies of advertising for attention, including studies of consideration set formation (Eliaz and Spiegler 2011), limited attention (Anderson and De Palma 2012, Hefti 2015), and brand prominence (Armstrong and Zhou 2011). Eliaz and Spiegler (2011) and Hefti (2015) study the effects of advertising (or other marketing strategies) in models with differentiated products.  In Eliaz and Spiegler (2011), all consumers begin with a status quo product, but may be persuaded to add an additional product to their consideration set.  In contrast to the baseline symmetric equilibrium result of Eliaz and Spiegler, in our framework, a firm's attempt to persuade customers to add its product to their consideration set is not necessarily effective and may not lead to a purchase even if it is.  This is in part driven by our focus on homogenous products, but also by the fact that, unlike Eliaz and Spiegler (2011), we envision a role for advertising strategy that is interdependent with pricing strategy. Indeed, one of our key interests concerns how advertising for consideration affects price, something that Eliaz and Spiegler (2011) do not study. 

As we do in our work, Hefti (2015) considers the interaction between pricing and advertising. In a framework in which consumers have limited information capacity, Hefti finds that in contrast to a setting with complete information, an increase in the number of firms has an ``inhibition effect'' that leads consumers to choose relatively worse products, outweighing potential gains from increased product diversity.\footnote{This is in contrast to, say, Falkinger (2008) or Anderson et al. (2015). In Falkinger (2008), consumers with limited attention benefit by moving to an ``information-rich'' economy in which firms vie for attention.  Unlike Hefti (2015) or our paper, Falkinger does not consider strategic interactions between prices and advertising. In the Anderson et al. (2015) model of personalized pricing and advertising, as the cost of advertising tends to zero, consumer surplus tends toward the first-best level as the weight that the consumer's top ranked firm places on its most competitive offer tends to one. In a related working paper, Hefti and Liu (2016) study how limited attention can alter the optimal strategies of firms who can target market to individual consumers.}  Thus, as is sometimes the case in our framework, lower advertising costs can diminish consumer surplus (and holding the number of firms fixed, raise firm profit). However, the impact is indirect. Firm prices remain constant in the cost of advertising and cheaper advertising leads to market entry that creates a mismatch between the consumer and her ideal variety. Moreover, in Hefti's framework, from firms' perspective, there is no sense of ``over-advertising'' as occurs in our model when advertising costs decline so substantially that firms fail to segment the market.

Anderson and De Palma (2012) are interested in information overload in a multi-sector economy where firms within each sector compete to sell a homogenous good and consumers have a limited amount of attention to be divided across all sectors of the economy.  As in our framework, price dispersion can occur within a sector because consumers are heterogeneous with respect to the size of their consideration sets.  However, unlike in our model, all firms within a sector run price ads and have an equal chance of entering the consideration set, they do not advertise to increase their chance of being considered relative to their rivals. Thus, in contrast to our model, an across-the-board decrease in the cost of advertising does not impact the price distribution; improvements in advertising capability are offset by a proportional rise in advertising that dissipates these improvements.\footnote{Anderson and De Palma (2012) builds on an earlier framework, Anderson and De Palma (2009), in which a consumer determines the optimal number of messages to examine from firms.  Firm profits are exogenous, so that as in Falkinger (2008), the interplay between price and advertisement is not examined.}

A number of authors have also modeled advertising for attention without explicitly studying consideration set formation. Advertising in such models takes various forms: it creates brand loyalty among price-conscious consumers (Chioveanu 2008) or among new customers who may have been previously unaware of the product or preferred a different one (Loginova 2009), or alternatively, it influences the order in which consumers who face search costs are likely to visit each firm (Haan and Moraga-Gonzalez 2011). Perhaps the most closely related such model is that of Armstrong and Zhou (2011), in which competitors pay an intermediary to make their firm prominent. 

In the equilibrium of Armstrong and Zhou (2011), all relatively uninformed consumers are effectively persuaded to buy from the firm that pays the highest commission to the intermediary and becomes most prominent. Although not explicitly a model of consideration set formation, mathematically, this is a knife's edge variant of our framework, in which a non-advertising firm still captures some relatively uninformed consumers. A major distinction between our model and that of Armstrong and Zhou is that whereas in our case, the cost or effectiveness of advertising determines the ultimate proportions of captives and shoppers, these proportions are exogenously set by Armstrong and Zhou and determine the equilibrium commission (i.e., advertising expenditures). Thus, in our model, firms over-advertise when the cost of doing so is relatively low, whereas in Armstrong and Zhou (2011), firms advertise too much when there are sufficient uninformed consumers, in which case there is a big payoff to capturing all of them by being prominent.	   




%%%%%%%%%%    ADVERTISING AND PRICING IN A DUOPOLY     %%%%%%%%%%%
\section{Consideration Sets, Advertising and Pricing in a Duopoly}\label{Duopoly}

Throughout most of this paper, unless stated otherwise, we focus our attention on a duopoly setup. Consider two identical firms, firms 1 and 2, who compete via prices to sell a homogenous good. Firms' marginal production costs are normalized to zero and there are no capacity constraints. On the other side of the market, there is a unit mass of identical consumers with unit demand and valuation $v>0$ for the good. Consumers restrict purchasing decisions to only those firms in their consideration set. 

In addition to setting prices, firms can choose to advertise. Following Janssen and Non (2008), we model advertising as an all-or-nothing decision. In particular, a firm can pay cost $A>0$ to run an advertisement that is viewed by the entire market. Under this simple advertising technology, firms can advertise with different intensities by engaging in mixed advertising strategies. Let $\alpha_{i}$ denote the probability that firm $i$ advertises. We can interpret $\alpha_{i}$ as the probability that any given consumer views firm $i$'s ad.\footnote{As Janssen and Non (2008) point out, this simple advertising technology is equivalent to a framework in which firms choose an advertising intensity indicating the fraction of consumers who are influenced by an advertisement (as in the informative advertising models of Butters 1977, Robert and Stahl 1993, and Stahl 1994). For additional detail, see footnote 5 in Janssen and Non (2005).} 

In the remainder of this section, we first lay out a tractable model of consideration set formation where the effects of advertising can be introduced in a convenient way. Next, we explain how firm advertising affects consideration sets. Then, we describe a baseline pricing equilibrium without advertising for comparison to the results with advertising. Finally, we characterize the unique symmetric equilibrium in the full model with advertising.



%%%%%%%%   CONSIDERATION SETS   %%%%%%%
\subsection{Consumer Consideration Set Formation}\label{Consideration}
To make purchasing decisions, consumers undertake a two-stage process in which they first hone in on a subset of firms and then choose their most preferred firm.  We refer to this subset of firms as a \textit{consideration set}.  Additionally, firms may influence consideration set formation by advertising. 

When forming their consideration sets, consumers make two independent attempts to evoke firms with replacement. We call each attempt an ``evocation.'' Absent advertising, firms have equal probabilities of being evoked on each attempt.  Thus, by construction, this process guarantees that all consumers always have at least one firm in their consideration set.

In the duopoly setup, each firm has probability 1/2 of being evoked on the first attempt. Because each evocation is performed with replacement, each firm again has probability 1/2 of being evoked on the second attempt. This implies that for any given consumer, her consideration set contains only firm 1 with probability 1/4, only firm 2 with probability 1/4, and both firms with probability 1/2. Consumers whose consideration set contains only a single firm are said to be ``captive'' to that firm. The remaining consumers whose consideration set consists of both firms ``shop around'' for the best deal and are henceforth referred to as ``shoppers''. 



%%%%%%   CONSIDERATION SETS AND ADVERTISING   %%%%%
\subsection{Firm Advertising and Consideration Sets}\label{AdsConsideration}

In the context of this framework, the only effect of advertisement is to influence the probability of consideration by consumers. Both firms' advertising decisions, $\alpha_1$ and $\alpha_2$, directly impact the probability of entering consumers' consideration sets. For $\alpha_{i} \in (0,1)$, firm $i$'s advertisement may or may not be seen by any given individual. In particular, there are four potential outcomes regarding ad viewing for any given consumer: \vspace{-1ex}

\begin{enumerate}
\item[i.] Both ads are observed with probability $\alpha_{1}\alpha_{2}$. \vspace{-1.0ex} 
\item[ii.] No ads are observed with probability $(1-\alpha_{1})(1-\alpha_{2})$. \vspace{-1.0ex}
\item[iii.] Only the ad of firm 1 is observed with probability $\alpha_{1}(1-\alpha_{2})$. \vspace{-1.0ex}
\item[iv.] Only the ad of firm 2 is observed with probability $\alpha_{2}(1-\alpha_{1})$. 
\end{enumerate}

To capture the idea of advertising for consideration, we suppose that a consumer who views an ad from a firm $i$ but does not view an ad from that firm's rival $j$ will evoke firm $i$ with a higher probability. In this case, the consumer views advertising firm $i$ as \textit{prominent}, and consequently, more likely to be evoked than firm $j$. In outcomes (i) and (ii) above, neither firm attains prominence and both are equally likely to become evoked,\footnote{Which also implies that in outcome (i), advertising expenditures are wasted.} whereas in outcomes (iii) and (iv), the firm whose ad is observed becomes prominent. In other words, even if both firms advertise, there may not be a firm that becomes prominent for any given consumer.
 
When a firm is prominent to a consumer, she evokes the prominent firm with probability $(1+\mu)/2$ during each evocation, and evokes the non-prominent rival with probability $(1-\mu)/2$, where $\mu \in (0,1)$. The parameter $\mu$ can therefore be interpreted as the fraction of consumer memory that a prominent firm ``steals'' from a non-prominent rival. Thus, the prominent firm expects to enter a consumer's consideration set with a higher probability, $(1+\mu)^{2}/4>1/4$, than when neither firm is prominent. Analogously, its non-prominent rival now expects to enter the consumer's consideration set with lower probability $(1-\mu)^{2}/4<1/4$.

Putting all of this together, when no firm is prominent, we expect that 1/4 of consumers are captive to firm 1, 1/4 of consumers are captive to firm 2, and 1/2 of consumers are shoppers. In contrast, when there is a prominent firm, its fraction of captives increases by $\Delta_{H}/4$, where $\Delta_{H}\equiv (1+\mu)^{2}-1=\mu(2+\mu)$. Similarly, the non-prominent rival loses $\Delta_{L}/4$ captive consumers to the prominent firm, where $\Delta_{L}\equiv 1-(1-\mu)^{2}=\mu(2-\mu)$. Moreover, whenever there is a prominent firm, the fraction of overall shoppers decreases by $\Delta_{S}/4$ where $\Delta_{S}\equiv 2-2(1-\mu)(1+\mu)=2\mu^{2}$. This means that a prominent firm effectively ``steals'' consumers from the overall base of shoppers and from its rival's base of captives. In other words, $\Delta_{H}=\Delta_{L}+\Delta_{S}$. We refer to $\Delta_{H}$, which is monotonically increasing in $\mu$, as the gain from becoming prominent.

Thus, for a given firm, when its rival does not advertise, the firm has an incentive to advertise with the aim of becoming prominent and increasing its expected fraction of captive consumers by $\Delta_{H}/4$. If, however, its rival does advertise, the firm still has an incentive to advertise to attempt to keep its rival from becoming prominent---preventing the expected loss of $\Delta_{L}/4$ captive consumers to its prominent rival. 



%%%%%%%   DUOPOLY GAME   %%%%%%%%%
\subsection{The Duopoly Game}\label{duopoly}

Within this framework, firms and consumers play the following game. First, firms simultaneously make pricing and advertising decisions. Next, consumers view ads (if any), and then form consideration sets. Finally, consumers observe the prices in their consideration sets and make a purchasing decision. 

A strategy for firm $i$ is then given by $\{F_i(p)\text{, }\alpha_i\}$, where $F_i(p)$ represents the (unconditional) probability that firm $i$ offers a price no higher than $p$, and $\alpha_i$ denotes the probability that firm $i$ advertises. 

Consumers behave optimally according to the following purchasing rule: if firm $i$ is the only firm in a consumer's consideration set, she purchases at price $p_{i}$ only if $p_{i}\leq v$; otherwise, if another firm $j$ is in the consideration set as well, she only purchases from firm $i$ if $p_{i}\leq p_{j}$, and not at all if $\min\{p_i\text{, }p_j\}>v$.\footnote{As we show in the proof of Lemma~\ref{support_advertising}, when both firms are in the consideration set, the tie-breaking rule in the event of a pricing tie does not matter.}  

Unless stated otherwise, we restrict attention to symmetric Nash equilibria, henceforth denoting firm strategies by $\{F(p)\text{, }\alpha\}$, where $F(p)$ is distributed over support $[\underline{p}\text{, }\bar{p}]$. The unique symmetric equilibrium of this model will depend on the cost of advertising $A$ and the gain from becoming prominent $\Delta_{H}$, which is fully characterized by $\mu$.



%%%%%%%   BASELINE   %%%%%%
\subsubsection{Baseline Equilibrium Without Advertising}\label{Baseline}

First, we briefly describe the case where advertising is not possible, providing a baseline to compare with the scenarios where advertising is possible. Thus, for this case, an equilibrium is a price distribution $F^b(p)$, where $F^b(p)$ is distributed over support $[\underline{p}^b\text{, }\bar{p}^b]$. 

%%%%%%%  LEMMA 1  %%%%%%%

\begin{lemma} \label{support_advertising}
The equilibrium price distribution (1) contains no atoms and (2) is bounded by $v$ from above.
\end{lemma}

The process of consideration set formation as explained in Section~\ref{Consideration} generates two groups of consumers, captives and shoppers, with consideration sets of size 1 and 2, respectively. If all consideration sets always contained both firms (as would happen if consumers could make unlimited evocations), firms would price at marginal cost and earn zero profit. A limit on the number of evocations results in a form of consumer heterogeneity similar to that found in Varian (1980), which leads to price dispersion. Firms have monopoly power over their captives, who purchase at any price no greater than $v$. However, firms also have an incentive to undercut their opponent's prices to compete for shoppers, all of whom will purchase at the lower price. Consequently, firms always run sales---playing mixed pricing strategies between $v$ and some lower bound $\underline{p}^b$.

The following proposition fully characterizes this baseline equilibrium.

%%%%%    PROPOSITION 1   %%%%%%%%%%
\begin{proposition} \label{PureAd}
In a duopoly equilibrium without advertising, both firms play mixed pricing strategies over $[\underline{p}^b\text{, }v]$, according to distribution $F^b(p)$: 

\vspace{-1.5ex}
\begin{equation} \label{distribution}
	F^b(p)=1-\dfrac{1}{2}\left(\dfrac{v}{p}-1\right) \vspace{-1.5ex}
\end{equation}
where $\underline{p}^b=v/3$. In equilibrium, firms' expected profits are characterized by $\pi^b=v/4$.    
\end{proposition}

\begin{proof}
In equilibrium, a firm must be indifferent between any price in its support. Therefore, for any $p$ on the interior of the support of $F^b$, it must be that  

\vspace{-1.5ex} 
\begin{equation} \label{prof_base}
	\pi^b(p)=\frac{p}{4}\left[1+2(1-F^b(p))\right]=\frac{v}{4}=\pi^b(v) \vspace{-1.5ex}
\end{equation}
Solving for $F^b(p)$ yields Equation~\eqref{distribution}. Setting $F^b(\underline{p}^b)=0$ gives lower bound $\underline{p}^b=v/3$. \qedhere  

\end{proof}

As a result of Lemma~\ref{support_advertising}, $F^b(p)$ represents the unique symmetric equilibrium price distribution in this baseline without advertising. Price $\underline{p}^b$ is the lowest sale price that consumers can expect in equilibrium. This price is proportional to the ratio of captive consumers (1/4) to the sum of captives and shoppers per firm (3/4). This sum of captives and shoppers is the highest market share that the firm can achieve and is attained at $\underline{p}^b$. Finally, expected profit is equal to revenue at the monopoly price, which is the consumer valuation times the firm's share of captive consumers. At the top of the firm price distribution, the firm sells only to captive consumers. At any other price, it sells a higher quantity at a lower price. 



%%%%%%%%%  DUOPOLY EQUILIBRIUM  %%%%%%%
\subsubsection{Equilibrium with Advertising}\label{Duopoly Equilibrium}

In the full game with advertising, a symmetric equilibrium is given by $\{F(p)\text{, }\alpha\}$, where $F(p)$ is distributed over support $[\underline{p}\text{, }\bar{p}]$. As in the baseline, firms play a mixed pricing strategy with no atoms in the support and $\bar{p}=v$.\footnote{As shown in the Appendix, the proof of Lemma \ref{support_advertising} can be extended to include this case.} Depending on the values of parameters $A$ and $\mu$, there is a unique symmetric equilibrium that entails one of the following: (1) no advertising, (2) advertising with probability one, or (3) advertising with probability $\alpha\in(0\text{, }1)$.\footnote{With a further partition that depends on $\mu$ and $A$.}  

First, suppose that neither firm advertises---so that both firms obtain baseline expected profits $\pi^b(v)=v/4$---and that firm $i$  charges price $v$, selling only to its captives. Intuitively, at the high price of $v$, advertising serves solely to increase the number of captive consumers by $\Delta_{H}/4$. If firm $i$ chooses to advertise, it will become prominent, gaining $\Delta_{H}/4$ captives and increasing expected profits by $\Delta_{H}v/4$, but incurring a cost of $A$. When $\Delta_{H}v/4<A$, however, the marginal increase in revenue, $\Delta_{H}v/4$, is smaller than the advertising expenditure $A$, and a firm finds it unprofitable to advertise even when its rival does not advertise.\footnote{As we show in the proof of Theorem~\ref{all_equilibria}, the profit from deviating to advertising is highest at $v$.} Because neither firm has an incentive to advertise, $\alpha=0$. As a result, the equilibrium price distribution, its lower bound and expected profits are the same as in the baseline case.

In contrast, suppose that $A<\Delta_{H}v/4$ and that firm $j$ advertises with probability one. It is a best response for firm $i$ to advertise as well, to prevent losing consumers to a prominent rival. When both firms advertise, neither firm is prominent, and consideration sets are identical to the baseline case. However, because each firm spends $A$ to advertise, expected profits equal to $\pi^b(v)-A=v/4-A$, strictly lower than in the baseline case. A sufficient condition for not deviating from advertising is $A<\Delta_{H}v/12$. Intuitively, when the firm prices at $\underline{p}^b=v/3$, it runs the biggest sale possible and expects to capture all shoppers.\footnote{As we show in the proof of Theorem~\ref{all_equilibria}, the profit from deviating to not advertising is highest at $\underline{p}^b$.} By foregoing advertising, it saves $A$, but loses $\Delta_{L}/4$ captive consumers and $\Delta_{S}/4$ shoppers to a now prominent rival. That is, by not advertising at $\underline{p}^{b}$, it loses $\underline{p}^b(\Delta_{L}+\Delta_{S})/4=\underline{p}^b\Delta_{H}/4$ in revenue. Consequently, for $A<\Delta_{H}v/12$, low cost advertising results in a type of prisoners' dilemma for the firms whereby both engage in wasteful advertisement spending. 

Finally, when $A \in (\Delta_{H}/12\text{, } \Delta_{H}/4)$, firms play mixed advertising strategies---allowing each firm to become prominent with a positive probability---and, in expectation, firms who do not advertise charge lower prices than firms who do advertise. Thus, when there is a prominent firm, firms segment the market by inducing shoppers to buy from the non-prominent firm and a larger fraction of captives to buy from the prominent firm. We say that advertising is now effective because market segmentation shifts the distribution of prices relative to the baseline distribution. Depending on the cost of advertising and the gains from prominence, effective advertising may or may not increase expected profits with respect to the baseline.

The following theorem fully describes the unique symmetric equilibrium.

%%%%%   THEOREM 1   %%%%%%%
\begin{theorem}\label{all_equilibria} 
The unique symmetric duopoly Nash equilibrium is characterized as follows: When $A>\Delta_{H}v/4$, firms advertise with probability zero ($\alpha=0$). When $A<\Delta_{H}v/12$, firms advertise with probability one ($\alpha=1$). Finally, when $A\in(\Delta_{H}v/12\text{, }\Delta_{H}v/4)$, firms play mixed advertising strategies where advertising either partially or fully substitutes for lower prices.
\end{theorem} 

\begin{figure}[!thb]
\center
  \includegraphics[width=0.85\textwidth]{eqspace_mu.eps}
  \caption{Equilibrium over parameter space $\mu\times A$}
  \label{eqsp}
\end{figure}

Figure~\ref{eqsp} visualizes Theorem~\ref{all_equilibria} over the $\mu\times A$ parameter space. The top left (white) region corresponds to the case with no advertising ($\alpha = 0$) whereas the bottom right (black) region corresponds to the case where firms advertise with probability one ($\alpha = 1$). The gray regions in between constitute the part of the parameter space where firms play mixed strategies in advertising. 

In the light-gray region, a firm that does not advertise will always set a lower price than an advertising rival. For this reason we say that advertising fully substitutes for lower prices. In contrast, in the dark-gray region, a firm that does not advertise will sometimes, but not always, price below the lowest price charged by an advertising firm. Because of this we say that advertising partially substitutes for lower prices. We will carefully derive the case where advertising fully substitutes for lower prices in Section~\ref{Effective Advertising} and leave the case where advertising partially substitutes for lower prices to the Appendix.\footnote{Both cases have similar properties, but the former is easier to interpret.}

Additionally, in the Appendix, we derive the no-deviation conditions that lead to pure strategy advertising specified in Theorem~\ref{all_equilibria} and rule out remaining equilibrium possibilities---such as, for instance, the possibility that advertising could complement lower prices, as happens in much of the literature on informative advertising (e.g., Butters 1977; Robert and Stahl 1993; and Stahl 2000).




%%%%%%   EFFECTIVE ADVERTISING DUOPOLY   %%%%%%%%%
\section{Effective Advertising Duopoly} \label{Effective Advertising}
In this section, we analyze our main results under duopoly competition---namely, a mixed advertising equilibrium in which both firms advertise with some positive probability less than one and prices without advertising are always lower than prices with advertising (i.e. advertising fully substitutes for lower prices). 

To clarify the analysis, following Janssen and Non (2008), we characterize this equilibrium using price distributions that are conditional on the firm's advertising strategy. Let $F^{a}$ and $F^{n}$ denote respective firm price distributions conditional on advertising and not advertising, with respective supports $[\underline{p}^{a}\text{, }\bar{p}^{a}]$ and $[\underline{p}^{n}\text{, }\bar{p}^{n}]$. The unconditional firm price distribution can now be written $F(p)=(1-\alpha) F^{n}(p)+\alpha F^{a}(p)$ over support $[\underline{p}\text{, } \bar{p}]$, where $\underline{p}=\min\{ \underline{p}^{n}\text{, }\underline{p}^{a}\}$ and $\bar{p}=\max\{ \bar{p}^{n}\text{, }\bar{p}^{a} \}$.  Thus, firm strategies are denoted by $\{F^{n}(p)\text{, }F^{a}(p)\text{, }\alpha\}$.

Using this notation, the case where advertising fully substitutes for lower prices is now characterized by $\underline{p}^{n}<\bar{p}^{n}=\underline{p}^{a}<\bar{p}^{a}=v$. For consistency of notation, we define $p^{*}\equiv\bar{p}^{n}=\underline{p}^{a}$. As we already know from Theorem~\ref{all_equilibria}, this requires $A\in(\Delta_{H}v/12\text{, }\Delta_{H}v/4)$, along with an additional condition on parameter values that will be described later in this section.



%%%%%%%%   DERIVATION   %%%%%%%
\subsection{Derivation of Effective Advertising Equilibrium}\label{Derivation}

When a given firm advertises, expected profit at price $p$, denoted by $\pi^{a}(p)$, depends on whether or not the rival advertises. With probability $\alpha$, the rival also advertises and neither firm is prominent. In this case, the market is divided according to the Section~\ref{Baseline} baseline whereby each firm sells to its respective captive consumers and both firms compete over shoppers. On the contrary, with probability $1-\alpha$, the rival does not advertise and the advertising firm becomes prominent, selling to a higher fraction of captives $(1+\mu)^2/4$. However, because a firm that advertises is underpriced for sure, it does not attract any shoppers. This yields

\vspace{-1.5ex}
\begin{equation} \label{prof1sub}
	\pi^{a}(p) = \underbrace{\alpha \frac{p}{4}\left[ 1+2(1-F^{a}(p)) \right]}_{\text{rival advertises}}+\underbrace{(1-\alpha)\frac{p}{4}(1+\mu)^2}_{\text{rival doesn't advertise}}-A   
\end{equation} 

In equilibrium, $\pi^{a}(v)=\pi^{a}(p)$ for $p\in [p^{*}\text{, }v)$. Solving this equation for $F^{a}(p)$ yields 

\vspace{-1.5ex}
\begin{equation} \label{F1sub}
	F^{a}(p)=1-\dfrac{(1+\mu)^2-\alpha\Delta_{H}}{2\alpha}\left(\dfrac{v}{p}-1\right) \vspace{-1.5ex}
\end{equation}

Setting $F^{a}(p^{*})=0$, we can solve for the lowest price with advertising, $p^{*}$, as a function of $\alpha$. Note that, by construction, $F^{a}(v)=1$.

Similarly, when a firm does not advertise, expected profit at price $p$, denoted by $\pi^{n}(p)$, still depends on the opponent's advertising strategy.  With probability $\alpha$, the rival advertises and becomes prominent. Here, the non-advertising firm sells to a lower fraction of captives $(1-\mu)^2/4$ but captures all shoppers $(1-\mu^2)/2$ that remain in the market. With probability $1-\alpha$, the rival does not advertise either, no firm is prominent, and both firms behave as in the Section~\ref{Baseline} baseline. Thus, expected profit is given by

\vspace{-1.5ex}
\begin{equation} \label{prof0sub}
	\pi^{n}(p)=\underbrace{\alpha \frac{p}{4}\left[(1-\mu)^2+2(1-\mu^2) \right]}_{\text{rival advertises}}+\underbrace{(1-\alpha) \frac{p}{4}\left[ 1+2(1-F^{n}(p)) \right]}_{\text{rival doesn't advertise}} 
\end{equation}

Setting $\pi^{n}(p^{*})=\pi^{n}(p)$ for $p\in [\underline{p}^{n}\text{, }p^{*})$ we can solve for $F^{n}(p)$ as a function of $p^{*}$.

\vspace{-1.5ex}
\begin{equation} \label{F0sub}
	F^{n}(p)=1-\dfrac{1-\alpha\left(\Delta_{H}-2\right)}{2(1-\alpha)}\left(\dfrac{p^{*}}{p}-1\right) \vspace{-1.5ex}
\end{equation}

We can set $F^{n}(\underline{p}^{n})=0$ to solve for the lowest price without advertising, $\underline{p}^{n}$, in terms of $\alpha$. Moreover, $F^{n}(p^{*})=1$ by construction.

To solve for the equilibrium probability of advertising $\alpha$, we can use the fact that in equilibrium, each firm is indifferent between advertising and not advertising at price $p^{*}$, such that $\pi^{a}(p^{*})=\pi^{n}(p^{*})$. This yields 

\vspace{-1.5ex}
\begin{equation} \label{asub}
	\alpha=\dfrac{(1+\mu)^2\left(\Delta_{H}v/4-A \right)}{\Delta_{H}\left(\Delta_{H}v/4-A \right)+2A} \vspace{-1.5ex}
\end{equation}
 
We need to verify that firms do not wish to deviate from their equilibrium strategy. First, when a firm advertises, it must not wish to deviate to a price below $p^{*}$.  Second, when a firm does not advertise, it must not wish to deviate to a price above $p^{*}$. Finally, firms must not want to deviate from their advertising strategy, $\alpha$.

To see the first condition, we can write down $\pi^{a}(p)$ for $p<p^{*}$.\footnote{In this case, $\pi^{a}(p) = 3\alpha p/4+(1-\alpha)p[(1+\mu)^2+2(1-\mu^2)(1-F^{n}(p))]/4-A$.} Using the value of $F^{n}(p)$ calculated in Expression~\eqref{F0sub}, we find that $\mathrm{d}\pi^{a}(p)/\mathrm{d}p>0$, so that it is unprofitable for the advertising firm to lower its price below $p^{*}$.  

Similarly for the second condition, we write $\pi^{n}(p)$ allowing for $p>p^{*}$.\footnote{In this case, $\pi^{n}(p) = \alpha p[(1-\mu)^2+2(1-\mu^2)(1-F^{a}(p))]/4+(1-\alpha)p/4$.} When $\pi^{n}(p)$ is decreasing in prices, a firm does not wish to charge prices above $p^{*}$. Thus, taking the derivative of $\pi^{n}(p)$ with respect to $p$ and substituting $F^{a}(p)$ from Expression~\eqref{F1sub}, we find that a necessary condition for $\mathrm{d}\pi^{n}(p)/\mathrm{d}p<0$ to hold is 

\vspace{-1.5ex}
\begin{equation} \label{cond_alpha_Reversed}
	\Delta_{H}-2\Delta_{S}\left[(1+\mu)^2-\alpha\left(\Delta_{H}-\frac{1}{2}\right)\right]>0   \vspace{-1.5ex}
\end{equation}

Substituting $\alpha$ from Expression~\eqref{asub} into Expression~\eqref{cond_alpha_Reversed}, the necessary condition becomes 

\vspace{-1.5ex}
\begin{equation} \label{condA_Reversed}
	A(\Delta_{H}-2) < \mu^{2}(2-\mu^{2})\frac{v}{4}  \vspace{-1.5ex}
\end{equation}

Condition~\eqref{condA_Reversed} always holds whenever $\mu<\sqrt{3}-1\approx 0.73$ because the left-hand side becomes negative. However, for higher values of $A$ and $\mu$, Condition~\eqref{condA_Reversed} fails and the non-advertising firm has incentives to increase prices above $p^{*}$.\footnote{When $\mu>\sqrt{3}-1$, the left-hand side is positive and Condition~\eqref{condA_Reversed} requires approximately $\mu<0.83$ for values of $A$ close to $\Delta_{H}v/4$, and holds even for values of $\mu$ very close to 1 for $A$ close to $\Delta_{H}v/12$. This corresponds to the dark-gray region of Figure~\ref{eqsp}.} Intuitively, because $\alpha$ is decreasing in $A$ in equilibrium,\footnote{As can be verified by taking the derivative of Expression~\eqref{asub} with respect to A.} Condition~\eqref{condA_Reversed} indicates that whenever the gain from becoming prominent is relatively high, the probability that a non-advertising firm's rival does advertise (and consequently sets a price above $p^{*}$) must be sufficiently high as to deter the non-advertising firm from pricing above $p^{*}$.

Lastly, a sufficient condition for $\alpha\in(0,1)$ is that $A\in\left(\Delta_{H}v/12,\Delta_{H}v/4\right)$. As we already explained in Section~\ref{Duopoly Equilibrium}, when $A>\Delta_{H}v/4$, firms want to deviate to not advertising at all ($\alpha=0$), and when $A<\Delta_{H}v/12$, firms want to deviate to advertising all the time ($\alpha=1$).\footnote{Together with Condition~\eqref{condA_Reversed}, this implies that this equilibrium falls into the light-gray region of Figure~\ref{eqsp}.} 

To obtain firms' equilibrium expected profit, we substitute $\alpha$ into $\pi^{a}(v)$. Proposition~\ref{Ad_Subs_Low_Price} summarizes the results obtained in this section.

%%%%%%%%   PROPOSITION 2   %%%%%%%%
\begin{proposition} \label{Ad_Subs_Low_Price}
In a mixed advertising strategy duopoly equilibrium in which advertising fully substitutes for lower prices, both firms play mixed pricing strategies according to $F(p)=(1-\alpha)F^{n}(p)+\alpha F^{a}(p)$, where $\alpha$ is given by Expression~\eqref{asub}, and $F^{n}(p)$ and $F^{a}(p)$ take values between zero and one on respectively $[\underline{p}^{n}\text{, }p^{*}]$ and $[p^{*}\text{, }v]$ according to 

\vspace{-1.5ex}
\begin{equation*}
	F^{n}(p)=1-\dfrac{1-\alpha\left(\Delta_{H}-2\right)}{2(1-\alpha)}\left(\dfrac{p^{*}}{p}-1\right)\text{,}\qquad F^{a}(p)=1-\dfrac{(1+\mu)^2-\alpha\Delta_{H}}{2\alpha}\left(\dfrac{v}{p}-1\right) \vspace{-1.5ex}
\end{equation*}

Additionally, $\underline{p}^{n}$ and $p^{*}$ are given by 

\vspace{-1.5ex}
\begin{equation*}
	\underline{p}^{n}=\dfrac{1-\alpha\left(\Delta_{H}-2\right)}{3-\alpha\Delta_{H}}p^{*} \text{,}\qquad p^{*}=\dfrac{(1+\mu)^2-\alpha\Delta_{H}}{(1+\mu)^2-\alpha(\Delta_{H}-2)}v \vspace{-1.5ex}
\end{equation*}

In equilibrium, expected profit equals $\pi=\left[(1+\mu)^2-\alpha\Delta_{H}\right]v/4- A$.
\end{proposition}

To recapitulate, for an intermediate range of $A$ and $\mu$, firms mix between advertising and not advertising such that sales ran by a non-advertising firm are always better than sales run by an advertising firm---that is, advertising fully substitutes for lower prices. 

As we discussed in Section~\ref{Introduction}, the equilibrium outcome in Proposition~\ref{Ad_Subs_Low_Price} contrasts the outcomes in most models of informative advertising, in which advertised prices are lower than unadvertised ones. Consider, for instance, the oft cited framework of Baye and Morgan (2001), in which a gatekeeper allows consumers to compare prices beyond their local monopoly.\footnote{This framework has been extended in multiple directions. For instance, Arnold et al. (2011) derive pricing distributions and comparative statics when firms start out with varying proportions of local consumers. Arnold and Zhang (2014) show that the symmetric equilibrium of Baye and Morgan (2001) is unique in a commonly used variant of that framework in which consumers do not pay for the gatekeeper's services. Shelegia and Wilson (2016) generalize the Baye and Morgan framework to allow for firm-level heterogeneity by recasting firms as competing in net utility rather than prices.\label{Baye_Morgan_lit}} In this model, a firm that does not advertise cannot attract consumers from outside its local market by charging a low price. In contrast, an advertising firm must compete with other advertising firms for consumers who subscribe to the information gatekeeper's service. Thus, advertising effectively makes an individual firm's demand more elastic, inducing it to lower its price. The opposite occurs in our framework because advertising diminishes the fraction of shoppers in the market. Moreover, as we discuss in the next section, unlike in Baye and Morgan (2001), firms' expected profits do not necessarily decline as advertising becomes more intense.\footnote{Although comparative statics concerning advertising differ in the models described in footnote~\ref{Baye_Morgan_lit}, the Baye and Morgan result that firms would have been better off not advertising at all persists in these models.}   



%%%%%%%%   CHARACTERIZATION   %%%%%%%%%
\subsection{Characterization of Effective Advertising Equilibrium} \label{Characterization of Duopoly Equilibrium}

In this section, we show that for certain parameter ranges, firms can increase prices and profits by segmenting the market via advertising. With this purpose, we proceed to explain the main features of the equilibrium derived in Section~\ref{Derivation}, including exploring the comparative statics with respect to parameters $A$ and $\mu$. Whether or not profits increase depends on the probability of advertising $\alpha$, which is driven by the cost of advertising and gain from becoming prominent when firms play mixed advertising strategies. Recall that the gain from becoming prominent, $\Delta_H$, is strictly increasing in $\mu$, so any comparative statics results with respect to $\mu$ also hold for $\Delta_H$. 


%%%%%%% PROBABILITY OF ADVERTISING   %%%%%%%
\subsubsection{Probability of Advertising} \label{Probability of Advertising}
Taking derivatives of the right-hand side of Expression~\eqref{asub} with respect to $A$ and $\mu$, it is easy to show that $\alpha$ is decreasing in $A$ and increasing in $\mu$. To gain some further intuition about these relationships, we rearrange Expression~\eqref{asub} as follows

\vspace{-1.5ex}
\begin{equation} \label{asub_alternative}
	\frac{A}{(\Delta_{H}/4)}=p^{*}=\dfrac{\alpha/4+(1-\alpha)(1+\mu)^2/4}{3\alpha/4+(1-\alpha)(1+\mu)^2/4}v \vspace{-1.5ex}
\end{equation}

Recall that the equilibrium value of $\alpha$ makes firms indifferent between advertising and not advertising at price $p^{*}$. When a non-advertising firm pricing at $p^{*}$ deviates to advertising, it incurs cost $A$ to gain $\Delta_{H}/4$ consumers, regardless of whether or not its opponent is advertising. Specifically, if its rival is advertising, the deviating firm prevents its advertising rival from becoming prominent and increases its fraction of captives by $\Delta_{L}/4$ and shoppers by $\Delta_{S}/4$.\footnote{Recall from Section~\ref{Consideration} that $\Delta_{H}=\Delta{L}+\Delta_{S}$.} Alternatively, if its rival is not advertising, the deviating firm becomes prominent and increases its fraction of captives by $\Delta_{H}/4$. 

Therefore, intuitively, the left-hand side of Equation~\eqref{asub_alternative} represents the cost of attracting an additional buyer via advertising at $p^{*}$, and the right-hand side represents the revenue from selling to an additional buyer. Unsurprisingly, as the cost of attracting an additional buyer increases, firms advertise with lower probability. From the left-hand side of Equation~\eqref{asub_alternative}, it is clear that the cost of attracting an additional buyer via advertising rises in $A$ and falls in $\mu$.


%%%%%%%   MARKET SEGMENTATION   %%%%%%
\subsubsection{Market Segmentation} \label{Market Segmentation}

When firms mix between advertising and not advertising, a firm can become prominent with positive probability. This, paired with its pricing strategy, allows each firm to cater to different groups of consumers. For this reason, we say that firms can use advertising to segment the market. 

In particular, with probability $\alpha(1-\alpha)$, firm 1 is prominent, advertising while its rival does not. Similarly, with probability $\alpha(1-\alpha)$ firm 2 is prominent. Together, with probability $2\alpha(1-\alpha)$, one of the firms is prominent and captures a greater number of captives than it does in the baseline. Furthermore, because the non-prominent firm underprices the prominent firm, the non-prominent firm captures all shoppers. However, with probability $\alpha^2+(1-\alpha)^2$, neither firm becomes prominent and both compete for shoppers, just as in the baseline.

Thus, in this mixed advertising strategy equilibrium, we say that advertising is effective because, with probability $2\alpha(1-\alpha)$, firms segment the market such that the prominent firm caters to a larger fraction of captives and the non-prominent firm serves all the shoppers. That is, advertising serves as a coordination device whereby firms will sometimes compete over the shoppers, and at other times ``cooperate'' to segment the market. This enables potential gains from not having to compete all the time. Going forward, we refer to $2\alpha(1-\alpha)$ as the probability of market segmentation. As we show, the higher the probability of market segmentation, the more likely that firms are able to raise prices by segmenting the market. 

Although profitable market segmentation is also found in the work of Janssen and Non (2008), as discussed in Section~\ref{Related Literature}, the result here is substantively different. In Janssen and Non (2008) because advertising is informative, a firm that advertises will always be considered by consumers. When successful market segmentation occurs in equilibrium, all consumers who were \textit{ex-ante} uninformed receive ads from the advertising firm and purchase from it at a higher price. In our model, when firms segment the market, the advertising firm will not always enter consumers' consideration sets. Such consumers evoke the non-advertising (non-prominent) firm twice and end up with the lower price in spite of being captive. This contrasts, Janssen and Non (2008), in which consumers who consider a single advertising firm always pay more than those who consider both firms. 


%%%%%%%%    FIRM PRICING    %%%%%%%%%    
\subsubsection{Firm Pricing} \label{Firm Pricing}

In order to compare expected prices when firms play mixed advertising strategies to those in the baseline, we first calculate expected prices in the baseline using $F^{b}$ from Proposition~\ref{PureAd}. This yields 

\vspace{-1.5ex}
\begin{equation} \label{expected_price_baseline}
	\operatorname{E}^b\left[p\right]=\int_{\underline{p}^b}^{v}{pdF^b(p)}=\frac{v}{2}\operatorname{ln}(3) \vspace{-1.5ex}
\end{equation}

By comparison, expected prices in the mixed advertising strategy equilibrium equal 

\vspace{-1.5ex}
\begin{equation} \label{expected_price_mixed}
	\operatorname{E}\left[p\right]=\alpha\int_{p^{*}}^{v}{pdF^{a}(p)}+(1-\alpha)\int_{\underline{p}^{n}}^{p^{*}}{pdF^{n}(p)} \vspace{-1.5ex}
\end{equation}   

Expression~\eqref{expected_price_mixed} is a complicated function of $A$ and $\mu$ that is difficult to evaluate analytically. As such, we compute it numerically for all values of $A$ and $\mu$ that fall into the light-gray region of Figure~\ref{eqsp}. Intuitively, it helps to analyze firm price distributions for both cases. The baseline price distribution, $F^b(p)$, is proportional to the ratio of a firm's captive consumers to shoppers, which equals 1/2. This is also the case when there is no prominent firm in the mixed advertising strategy equilibrium. However, when there is a prominent firm in the mixed advertising equilibrium, three things occur with respect to the baseline: (1) the prominent firm's fraction of captives increases (2) the overall fraction of shoppers decreases, and (3) the remaining shoppers are all served by the non-prominent firm. Thus, when there is market segmentation, firms are able to raise prices. In fact, the higher the probability of market segmentation, the more often firms can charge higher prices.  

In order to understand the relationship between expected prices and parameters $\mu$ and $A$, it is convenient to understand how these parameters affect the probability of market segmentation through their effect on $\alpha$. The probability of market segmentation, $2\alpha(1-\alpha)$, is maximized at $\alpha=1/2$.\footnote{Likewise, the corresponding probability, $\alpha^2+(1-\alpha)^2$, is minimized.} Moreover, recall that $\alpha$ is strictly increasing in $\mu$ and strictly decreasing in $A$. This means that, for a given value of $A$, the relationship between the probability of market segmentation and $\mu$ is shaped like an inverted U, and the same is true for the relationship between the probability of market segmentation and $A$ for a fixed value of $\mu$. When $\mu$ and $A$ are too low or too high, there is a lower probability that either firm becomes prominent, intensifying price competition over shoppers.   

\begin{figure}[!htb]
\center
  \includegraphics[width=0.48\textwidth]{2firms_price.eps}
  \includegraphics[width=0.48\textwidth]{2firms_profit.eps}
  \caption{Expected prices and profits as a function of $\mu$ and $A$}
  \label{EpandPi}
\end{figure} 

The left-hand panel of Figure~\ref{EpandPi} (calculated with $v$ normalized to one) shows that, when advertising fully substitutes for lower prices, expected price $\operatorname{E}\left[p\right]$ is non-monotonic in $\mu$ and $A$, and always greater than in the baseline.\footnote{In order to appropriately interpret the domain in Figure~\ref{EpandPi}, trace a horizontal line through Figure~\ref{eqsp} at $A=0.05$. Observe that $\alpha$ ranges from 0 (no advertising) at approximately $\mu=0.1$, to 1 (always advertising) at approximately $\mu=0.24$, spanning the domain of Figure~\ref{EpandPi}. As the legend in Figure~\ref{EpandPi} shows, thicker lines represent $\alpha$ for higher values of $A$.}\footnote{Additionally, we can show that the expected price conditional on advertising in this mixed advertising strategy equilibrium is always greater than $\operatorname{E}^b\left[p\right]$, whereas the expected price conditional on not advertising is always lower. A Mathematica file used for these computations is available upon request from the authors.}


%%%%%%%%   EXPECTED PROFITS   %%%%%%%
\subsubsection{Expected Firm Profits} \label{Firm Profits}

In Section~\ref{Firm Pricing}, we showed that expected prices are higher with respect to the baseline, however, this is not always the case for expected profits. Whereas, with market segmentation, advertising firms may be able to raise prices, they must also incur cost $A$. Therefore, advertising is more profitable than the baseline as long as the cost of advertising $A$ is low enough to make advertising worthwhile, but also sufficiently high to keep the probability of market segmentation high enough. Additionally, the gain from becoming prominent must not be so high as to induce excessive advertising. 

The intuition is similar to that for expected prices. Because the probability of market segmentation $2\alpha(1-\alpha)$ is maximized at $\alpha=1/2$, for a given value of $A$, the relationship between the probability of market segmentation and $\mu$ is shaped like an inverted U. For values of $\mu$ that induce firms to advertise ``too often,'' the benefits from market segmentation are outweighed by the decreased probability that segmentation actually occurs. When this happens, both firms pay $A$ but rarely become prominent and expected profits can even fall below the baseline level. 

For a given value of $\mu$, $A$ has direct and indirect effects on expected profits. While $A$ directly decreases profits, it indirectly lowers the probability of even incurring that cost by lowering $\alpha$. The right-hand panel in Figure~\ref{EpandPi} suggests that the direct effect dominates for lower values of $\mu$ (and expected profits decrease as $A$ increases), whereas the indirect effect dominates for higher values of $\mu$ (and expected profits increase as $A$ increases).

\begin{figure}[!thb]
\center
  \includegraphics[width=0.85\textwidth]{eqsp_banana.eps}
  \vspace{-1.5ex}
  \caption{Parameter region where expected profit is higher than in the baseline case}
  \label{eqspace1}
\end{figure}

The shaded area with dashed lines in Figure~\ref{eqspace1} represents the region of the $\mu\times A$ parameter space in which a mixed advertising strategy equilibrium is more profitable than the baseline. As the discussion in this section suggests, $\mu$ must be low enough to keep $\alpha$ sufficiently low for mixed advertising strategies to be profitable, and for a given $\mu$, $A$ must be neither too high nor too low.

 


%%%%%%%   OLIGOPOLY   %%%%%%%%%%%%%
\section{Effective Advertising Oligopoly}\label{Oligopoly}

In this section, we assess the robustness of the effective advertising equilibrium described above to an oligopoly setting with $N>2$ identical firms. Because our primary interest is on whether effective advertising can raise profits, we do not attempt to characterize the entire parameter space, instead focusing on whether the result and intuition from Section~\ref{Effective Advertising} persists and how it changes by varying the number of firms.



%%%%%%%%   CONSIDERATION SETS   %%%%%%%%%%%%
\subsection{Consideration Sets and Prominence}

Conceptually, consideration sets are still formed in a similar fashion to that described in Section~\ref{Consideration}. Although, to keep the model tractable, we assume that consumers make two evocations, so that consideration sets continue to consist of at most two firms, certain features of the consideration set formation process will have to be adjusted.  

First, we extend the idea of prominence to more than two firms by saying that firm $i$ is \textit{prominent} to a given consumer when that consumer views firm $i$'s ad and, additionally, there exists at least one firm $j$ whose ad the consumer does not view. When no firm is prominent, firms are evoked with equal probability $1/N$. Consequently, each firm anticipates serving $1/N^2$ captives and entering the consideration set of $2(N-1)/N^2$ shoppers. Note that, in this case, although the shoppers no longer consider every firm in the market, they still consider more firms than the captives do.\footnote{In other words, our framework is not equivalent to the Varian (1980) N-firm setup.} In this oligopoly setup, we refer to only those shoppers who evoke firm $i$ into their consideration set as the shoppers of firm $i$.

Suppose instead that $K\in(0,N)\cap\mathbb{N}$ firms are prominent to a consumer. In each attempt, she evokes a prominent firm into her consideration set with probability $\gamma(K)/N>1/N$, where $\gamma(K)=[K+\mu(N-K)]/K$, and a non-prominent firm with probability $(1-\mu)/N<1/N$. Here, as in the duopoly case, the parameter $\mu \in (0\text{, } 1)$ represents the fraction of consumer memory that prominent firms ``steal'' from non-prominent rivals. However, now this ``stolen memory''  (captured by $\mu(N-K)$) must be split among all prominent firms according to $\gamma(K)$. Intuitively, the higher the number of prominent firms, the smaller the total amount of memory that can be stolen from non-prominent rivals, and further, this smaller amount of available memory must also be split among more firms. Together, this helps to illustrate that $\gamma(K)$ is decreasing in the number of prominent firms $K$, and can be interpreted as the return to becoming prominent. 

Given the above, each prominent firm expects to serve $[\gamma(K)]^{2}/N^{2}$ captives, whereas a non-prominent firm expects $(1-\mu)^{2}/N^{2}$ captives. Because shoppers now only have two firms in their consideration sets, the composition of a firm's shoppers will also vary. In particular, for a given firm $i$, its expected fraction of shoppers depends on the other firm it is ``competing'' with in each of its shoppers' consideration sets. We can distinguish two cases: (1) when firm $i$ is prominent, and (2) when firm $i$ is not prominent. First, suppose that $K$ firms are prominent, including  firm $i$. The expected fraction of shoppers who observe firm $i$ plus another prominent firm is $2(K-1)[\gamma(K)]^2/N^{2}$, whereas the expected fraction of shoppers who observe firm $i$ plus a non-prominent firm is $2(N-K)(1-\mu)\gamma(K)/N^{2}$. Alternatively, suppose that $K$ firms are prominent, but firm $i$ is not prominent. Then, the expected fraction of shoppers who observe firm $i$ plus a prominent firm is $2K(1-\mu)\gamma(K)/N^{2}=2(1-\mu)[K+\mu(N-K)]/N^{2}$, whereas the expected fraction of shoppers who observe firm $i$ plus another non-prominent firm is $2(N-1-K)(1-\mu)^{2}/N^{2}$.  



%%%%%%%   EFFECTIVE ADVERTISING EQUILIBRIUM   %%%%%%
\subsection{Effective Advertising Equilibrium}

Following the notation from Section~\ref{Effective Advertising}, we construct the effective advertising equilibrium $\{F^{n}(p)\text{, }F^{a}(p)\text{, }\alpha\}$ where firms advertise with probability $\alpha \in (0,1)$. Advertising firms charge prices $p \in [p^{*},v]$\footnote{As in Section~\ref{Effective Advertising}, for consistency, we let $p^{*}\equiv\underline{p}^{a}=\bar{p}^{n}$. Additionally, going forward, we write $\underline{p}^{n}=\underline{p}$, where $\underline{p}<p^{*}<v$.} according to conditional price distribution $F^{a}(p)$, whereas non advertising firms charge prices $p \in [\underline{p},p^{*}]$ according to $F^{n}(p)$, where $F^{a}(p)$ and $F^{n}(p)$ are continuous and smooth over their domains. That is, advertising fully substitutes for lower prices.

Let $\pi^{a}(p)$ denote the expected profit of advertising firm $i$. With probability $\alpha^{N-1}$, no firm becomes prominent. When this is the case, firm $i$ sells to its captives and sells to its shoppers whenever it underprices the other firm that these shoppers evoked. With probability $(1-\alpha)^{N-1}$, firm $i$ becomes prominent, catering only to its captives because it will be underpriced by all the other competitors. Moreover, because firm $i$ is the only prominent firm in this case, it serves the largest possible fraction of captives, $\gamma[(1)]^{2}/N^{2}$. Finally, for $K<N-1$, $K+1$ firms (including firm $i$) become prominent with probability $\alpha^{K}(1-\alpha)^{N-1-K}$, each serving fraction of captives $\gamma(K+1)^{2}/N^{2}$. Here, firm $i$ may serve some of its shoppers as long as their consideration sets contain another prominent rival and firm $i$ offers the lower price. Otherwise, when firm $i$'s shoppers' consideration sets contain a non-prominent rival $j$, they purchase from firm $j$ with certainty. Thus, for every $p \in[p^{*},v]$, $\pi^{a}(p)$ is given by: 

\vspace{-1.5ex}
\begin{equation}\label{pi1}
	\begin{array}{l}
		\displaystyle{\pi^{a}(p)=\frac{p}{N^{2}} \Bigg\{ (1-\alpha)^{N-1}[\gamma(1)]^{2}+\alpha^{N-1}\left[1+2(N-1)\left(1-F^{a}(p)\right)\right]} \\
		\displaystyle{+\sum\limits_{k=1}^{N-2}\binom{N-1}{k}\alpha^{k}(1-\alpha)^{N-1-k}[\gamma(k+1)]^{2}\left[1+2k\left(1-F^{a}(p)\right)\right]\Bigg\} -A}
	\end{array} 
\end{equation}

Similarly, suppose that firm $i$ is not advertising and let $\pi^{n}(p)$ denote firm $i$'s expected profit. With probability $\alpha^{N-1}$, all firms except firm $i$ become prominent. Aside from serving its captives, firm $i$ caters to all of its shoppers for sure, as it underprices all of its rivals. With probability $(1-\alpha)^{N-1}$, no firm is prominent and firm $i$ sells to its captives and competes on price over its shoppers with the other firm that its shoppers evoked. Finally, with probability $\alpha^{K}(1-\alpha)^{N-1-K}$, $K<N-1$ rivals become prominent (while $N-1-K$ rivals do not). In this case, firm $i$ sells for sure to those of its shoppers who additionally evoked a prominent rival into their consideration sets, but competes on price over those who additionally evoked a non-prominent rival. Thus, for every price $p \in[\underline{p},p^{*}]$, $\pi^{n}(p)$ is given by: 

\vspace{-2.5ex}  
\begin{equation}\label{pi0}
	\begin{array}{l}
		\displaystyle{\pi^{n}(p)= \frac{p}{N^{2}}\Bigg\{(1-\alpha)^{N-1}\left[1+2(N-1)\left(1-F^{n}(p)\right)\right]} \\
		\displaystyle{+\alpha^{N-1}\left[(1-\mu)^{2}+2(N-1)(1-\mu)\gamma(N-1)\right]+\sum\limits_{k=1}^{N-2}\binom{N-1}{k}\alpha^{k}(1-\alpha)^{N-1-k}} \\
		\displaystyle{\times\left\{(1-\mu)^{2}\left[1+2(N-1-k)\left(1-F^{n}(p)\right)\right]+2k(1-\mu)\gamma(k)\right\} \Bigg\}}
	\end{array} 
	\vspace{-1.5ex}
\end{equation}

Following the same procedure used to solve for equilibrium in Proposition~\ref{Ad_Subs_Low_Price}, in the Supplemental Appendix, we use expected profit functions~\eqref{pi1} and \eqref{pi0} to solve for the equilibrium conditional price distributions and probability of advertising.


%%%%%   GAINS FROM EFFECTIVE ADVERTISING   %%%%%%
\subsection{Gains from Effective Advertising}

In this section, we focus on whether and when firms could gain from market segmentation relative to an equilibrium without advertising when $N>2$. In a baseline where firms do not advertise---similar to that of Section~\ref{Baseline} for the duopoly case---expected profits for each firm equal $v/N^{2}$.\footnote{This is easy to see using the same procedure used to solve for equilibrium in Proposition~\ref{PureAd}.} Subtracting this from profits in Expression~\eqref{pi1} evaluated at $v$, we find that firms can benefit from advertising whenever the following condition holds:\footnote{In simplifying Expression~\eqref{banana}, we use $(1-\alpha)^{(N-1)}+\alpha^{(N-1)}+\sum\limits_{k=1}^{N-2}\binom{N-1}{k}\alpha^{k}(1-\alpha)^{N-1-k}=1$.} 

\vspace{-2.5ex}
\begin{equation} \label{banana}
	\frac{v}{N^{2}} \left\{ (1-\alpha)^{N-1}\left[[\gamma(1)]^{2}-1\right]+\sum\limits_{k=1}^{N-2}\binom{N-1}{k}\alpha^{k}(1-\alpha)^{N-1-k}\left[[\gamma(k+1)]^{2}-1\right] \right\} >A
 	\vspace{-1.5ex}
\end{equation}

Intuitively, a firm can benefit from advertising as long as it can attain prominence. Given opponents' equilibrium advertising strategy $\alpha$, with probability $\alpha^{N-1}$ no firm becomes prominent and advertising is wasteful. In contrast, a firm attains its largest customer gain when it becomes the only prominent firm, which occurs with probability $(1-\alpha)^{N-1}$. Attaining prominence thus increases the firm's fraction of captives by $\left\{[\gamma(1)]^{2}-1\right\}/N^{2}$. In general, for any number of prominent rivals $K>0$, the gain from becoming prominent is $\left\{[\gamma(K+1)]^{2}-1\right\}/N^{2}$, which is decreasing in $K$. As long as the expected gain from becoming prominent (the left-hand side of Inequality~\eqref{banana}) is greater than the cost of engaging in advertising ($A$), firms can benefit by using advertising to segment the market, just as in the duopoly case. It must be noted that the left-hand side of Inequality~\eqref{banana} also depends on the cost of advertising indirectly through $\alpha$, which is a complicated function of $A$, $\mu$, and $N$ that we solve for implicitly in the Supplemental Appendix.

The light-gray areas in Figure~\ref{eqspaceN} represent the regions of the $\mu\times A$ parameter space in which a mixed advertising strategy equilibrium is more profitable than the baseline (i.e., Inequality~\eqref{banana} holds) for different values of $N$.\footnote{Because when $N>2$, we could only solve for $\alpha$ implicitly, in order to graph Figure~\ref{eqspaceN}, using a large set of sample values of $\alpha\text{, }\mu\in(0\text{, }1)$, we computed the value of $A$ corresponding to each $\alpha$ and $\mu$ and then calculated profits for this set of values.} Dark-gray regions represent alternative equilibria which we have not characterized. Observe that, for greater values of $N$, the light-gray region takes up a greater proportion of the $\mu\times A$ parameter space.\footnote{This does not imply that advertising firms earn more profit when $N$ is higher. In fact, for a given value of $v$, expected profits in the baseline decrease in the number of firms $N$.}

\begin{figure}[H]
\center
  \includegraphics[width=0.48\textwidth]{3Firms.eps}
	\includegraphics[width=0.48\textwidth]{4Firms.eps}
	\includegraphics[width=0.48\textwidth]{7Firms.eps}
  \includegraphics[width=0.48\textwidth]{12Firms.eps}
  \caption{Parameter regions where expected profit is higher than in the baseline case}
  \label{eqspaceN}
\end{figure} 


%%%%%%%   MARKET SEGMENTATION   %%%%%%%
\subsection{Market Segmentation}

Finally, we can compute the probability of market segmentation as $1-\alpha^{N}-(1-\alpha)^{N}$. Just as in Section~\ref{Market Segmentation}, market segmentation comprises of all the cases where there are prominent firms and excludes the cases where no firm is prominent. In general, the number of firms $N$ has an ambiguous effect on the probability of market segmentation. Because $\alpha\in(0\text{, }1)$, $1-\alpha^{N}-(1-\alpha)^{N}$ is increasing in $N$. However, the equilibrium value of $\alpha$ also changes with $N$ for a given $\mu$ and $A$. As Table~\ref{table:effective_ad} indicates for $A=0.15$, and $N=2\text{, }3$, this depends on whether an increase in $N$ moves $\alpha$ closer to or further from $\alpha=1/2$ (the value that maximizes the probability of market segmentation). 

\begin{table}[ht] \label{effective_ad}
\caption{Probability of Market Segmentation ($A=0.15$)} 
\centering 
\begin{tabular}{c c c c} 
\hline 
\hline 
\multicolumn{1}{c}{$\mu=0.33$}\\
\hline
$N$ &  $\alpha$ & Prob. Market Segmentation & Expected Profit\\
\hline
 2 & 0.22 & 0.34 & 0.25\\
 3 & 0.11 & 0.29 & 0.12\\
\hline
\hline
\multicolumn{1}{c}{$\mu=0.56$}\\
\hline
 $N$ &  $\alpha$ & Prob. Market Segmentation & Expected Profit\\
\hline
 2 & 0.84 & 0.27 & 0.16\\
 3 & 0.49 & 0.75 & 0.10\\
\hline
\hline
\end{tabular}
 \vspace{-1.5ex}
\label{table:effective_ad}
\end{table} 




%%%%%%%     C O N C L U S I O N      %%%%%%%%
\section{Conclusion} \label{Conclusion}  

We have analyzed an intuitive and highly tractable model where firms compete to enter consumers' consideration sets via advertising. In this context, advertising can serve a role beyond information transmission or persuasion, serving instead to influence the chance of consideration by consumers. This type of advertising is particularly important because of the decline of traditional (television based) advertising, and growth of uninformative advertising through such means as product placement, which is intended to get at consumer mindsets (Lee and Faber 2007), but does not necessarily change individuals' attitudes toward a product or brand (Williams et al. 2011). Our model allows us to analyze this feature of advertising.

Using our framework, we show how the cost and effectiveness of advertising for attention impact prices and profits. Interestingly, if advertisements are able to make a firm overly prominent, firms advertise too much, such that, even if advertising for attention is a low cost activity, it remains unprofitable. The impact of cost is more nuanced. If advertising is inexpensive, the effect is similar to that when advertising is capable of making a firm very prominent---firms advertise too much and advertising is wasteful. If the cost of advertising is high enough, firms may be able to profit by segmenting the market because the probability that both firms run ads is sufficiently low. However, if the cost of advertising is too high, it can eat into profits and can even keep firms from engaging in advertising at all.

With global advertising spending projected to surpass \$550 billion U.S. dollars in 2018, advertising costs are not keeping firms from advertising.\footnote{Global advertising is projected to reach \$557.99 billion U.S. dollars in 2018, growing by 4.3 percent over the prior year. See Statista, The Statistics Portal. ``Global advertising spending from 2010 to 2017 (in billion U.S. dollars). Available at \href{https://www.statista.com/statistics/236943/global-advertising-spending/}{https://www.statista.com/statistics/236943/global-advertising-spending/}. Visited March 29, 2018.} However, ad spending is not dispositive of competition in the advertising market. As such, policy makers should bear in mind that increases in the costs of advertising brought about by the diminution of advertising competition, such as through mergers between large advertising platforms, could raise downstream retail prices by means other than cost pass-through. As we show, increased advertising costs could lead to price increasing co-opetition via market segmentation. Furthermore, these price increases may not even translate to greater profits for advertising firms, making both consumers and advertisers worse off. 

Our results concerning the price and profit-enhancing features of advertising are robust to various alternative formulations of our model. We find that, in an oligopoly setup, an effective advertising equilibrium with similar properties to the one in the duopoly setup exists, and leads to higher profits over a larger part of the parameter space than it does in a duopoly when compared to the no-advertising counterfactual. In the Supplemental Appendix, we develop two additional extensions where the market segmentation result becomes slightly weaker. In the first one, we consider sequential advertising and pricing decisions, and in the second we consider firms facing different costs of advertising. In both cases, the results involve asymmetric equilibria that still show a flavor of market segmentation. Although firms no longer exclusively focus on different consumer segments, in equilibrium, one firm focuses more on pricing to captives, whereas the other focuses more on shoppers relative to its rival.  




%%%%%%%%     APPENDIX     %%%%%%%%%%
\section*{Appendix}

\noindent \textbf{Proof of Lemma \ref{support_advertising}}

\begin{proof}
\textcolor{white}{.} 
\begin{itemize} 
\item[1.] Suppose that both firms have an atom at $p\leq v$, and that neither firm $i$, nor firm $j\neq i$ advertises. Firm $i$'s expected profit at $p$ is $p\left(1/4 +\tau_i/2\right)$, where $\tau_i$ is the proportion of shoppers who buy from firm $i$ when both firms charge the same price. Suppose instead that firm $i$ sets $p_i=p-\varepsilon$, for some small $\varepsilon>0$. Firm $i$'s profits become $3(p-\varepsilon)/4$, which is strictly higher for $\varepsilon$ sufficiently small and $\tau_i\in[0\text{, }1)$.

Suppose that firm $j$ chooses a price other than $p$.  Lowering the price charged never reduces the number of sales, so the loss to firm $i$ from lowering the price by $\varepsilon$ is at most $\varepsilon$. However, when $p$ is charged with positive probability, lowering the price by $\varepsilon$ will with positive probability lead to a gain and with complementary probability, at worst lead to a loss of $\varepsilon$. Therefore, by shifting its atom from $p$ to $p-\varepsilon$ for sufficiently small $\varepsilon$, firm $i$ increases its expected profit, a contradiction. If instead, $\tau_i=1$, then although firm $i$ will not wish to undercut an atom at $p$, firm $j$ will. This finishes the proof of part (1) for the baseline case.

For the case with advertising required for Theorem~\ref{all_equilibria}, add the following paragraph. Next, suppose that firm $j$ did advertise. Then firm $i$'s expected profit at $p$ equals $p\left[(1-\mu)^2/4 +\tau_i (1-\mu^2)/2\right]$, whereas at $p_i=p-\varepsilon$ the expected profit is $(p-\varepsilon)\left[(1-\mu)^2/4\right.$ $\left.+ (1-\mu^2)/2\right]$, which is strictly higher for $\varepsilon$ sufficiently small. Thus, firm $i$ wishes to undercut an atom at $p$ even if firm $j$ advertises. This logic extends to advertising mixtures by firm $j$. The proof follows similarly if firm $i$ did advertise. 

\item[2.] Because there are no atoms in the price distribution, we know that firms must be playing mixed strategies according to some distribution $F$. Let $\bar{p}$ be the upper bound of the support of $F$. A firm pricing at $\bar{p}>v$ makes no profit. Suppose instead that $\bar{p}<v$. Because there are no atoms, a firm pricing at $\bar{p}$ will be underpriced for sure and sell only to its captive consumers. Such a firm can increase its profit by raising its price to $v$.   \qedhere
\end{itemize}
\end{proof}




\noindent \textbf{Proof of Theorem \ref{all_equilibria}}
\begin{proof}
\textcolor{white}{.} 
In this proof, we first provide the no-deviation conditions for the pure advertising strategy cases. We then characterize the equilibrium where advertising partially substitutes for lower prices discussed at the end of Section~\ref{Duopoly Equilibrium} (dark-gray region of Figure~\ref{eqsp}). The equilibrium where advertising fully substitutes for lower prices is characterized in Section~\ref{Effective Advertising}. Finally, we proceed to rule out other potential equilibria in mixed strategies. \vspace{1.5ex}

\noindent \textbf{No-deviation conditions:}  

Suppose that $\alpha=0$. Both firms expect to earn expected profits $v/4$. If deviating to advertising, a firm earns $(p/4)[(1+\mu)^2+2(1-\mu^2)(1-F^{b}(p))]-A$, which is increasing in $p$ up to $v$. Therefore, the no-deviation condition is  

\vspace{-1.5ex}
\begin{equation}
	\frac{v}{4}>\frac{v(1+\mu)^2}{4}-A \Leftrightarrow A>\frac{\Delta_{H}v}{4}  \vspace{-1.5ex}
\end{equation} 

Similarly, if $\alpha=1$, both firms expect to earn expected profits $v/4-A$. By deviating to not advertising, a firm earns $(p/4)[(1-\mu)^2+2(1-\mu^2)(1-F^{b}(p))]$, which is decreasing in $p$ starting at $\underline{p}$. The no-deviation condition is 

\vspace{-1.5ex}
\begin{equation}
	\frac{v}{4}-A>\frac{\underline{p}}{4}\left[(1-\mu)^2+2(1-\mu^2)\right] \Leftrightarrow \frac{\Delta_{H}v}{12}>A  
\end{equation} 

\noindent \textbf{Equilibria where ads partially substitute low prices ($\boldsymbol{\bar{p}^{a}=\bar{p}^{n}=v}$):} 

When $\bar{p}^{a}=\bar{p}^{n}=v$, a firm that advertises expects profit at price $p\in[\max\{\underline{p}^{n},\underline{p}^{a}\}\text{, }v]$ of 

\vspace{-1.5ex}
\begin{equation} \label{prof1sub2}
	\pi^{a}(p) = \frac{p}{4}\left\{\alpha\left[ 1+2(1-F^{a}(p)) \right]+(1-\alpha)\left[(1+\mu)^{2}+2(1-\mu^{2})(1-F^{n}(p))\right]\right\}-A \vspace{-1.5ex}
\end{equation}

As was the case in Expression~\eqref{prof1sub}, with probability $\alpha$, neither firm is prominent and firms split the market. However, in contrast to the equilibrium characterized in Section~\ref{Effective Advertising}, in this case, if the rival does not advertise, a prominent firm is not certain to be underpriced and may still attract the remaining $(1-\mu^{2})/2$ shoppers with probability $1-F^{n}(p)$. 

If instead a firm does not advertise, its expected profit at price $p$ is given by 

\vspace{-1.5ex}
\begin{equation} \label{prof0sub2}
	\pi^{n}(p) = \frac{p}{4}\left\{\alpha\left[(1-\mu)^{2}+2(1-\mu^{2})(1-F^{a}(p)) \right]+(1-\alpha)\left[1+2(1-F^{n}(p))\right]\right\} \vspace{-1.5ex}
\end{equation}

Unlike in Expression~\eqref{prof0sub}, even if the rival advertises, when a firm that does not advertise prices in $[\max\{\underline{p}^{n},\underline{p}^{a}\}\text{, }v]$, with probability $F^{a}(p)$ it is underpriced, leaving it with no shoppers and fewer captives than a prominent rival.

Setting $\pi^{n}(p)=\pi^{n}(v)$ and $\pi^{a}(p)=\pi^{a}(v)$ gives us a system of two equations in two unknowns that solves for $F^{n}(p)$ and $F^{a}(p)$ on the interval $[\max\{\underline{p}^{n},\underline{p}^{a}\}\text{, }v]$ as follows. 

\vspace{-1.5ex}
\begin{equation} \label{F0sub2}
	F^{n}(p)=1+\dfrac{2\Delta_{H}-\Delta_{S}\left[(1+\mu)^{2}-\alpha\left(\Delta_{H}-2\right)\right]}{(1-\alpha)\Delta_{S}(4-\Delta_{S})}\left(\dfrac{v}{p}-1\right) 
	\vspace{-1.5ex}
\end{equation}

\begin{equation} \label{F1sub2}
	F^{a}(p)=1-\dfrac{\left(2\Delta_{H}+\Delta_{S}\right)-\alpha\Delta_{S}\left(\Delta_{L}+2\right)}{\alpha\Delta_{S}(4-\Delta_{S})}\left(\dfrac{v}{p}-1\right) 
\end{equation} 

In equilibrium, it is possible that either $\underline{p}^{n}$ or $\underline{p}^{a}$ is higher. Suppose first that $\underline{p}^{a}\geq\underline{p}^{n}$. Setting $F^{a}(\underline{p}^{a})=0$ yields the lowest price charged with advertising, $\underline{p}^{a}$.

\vspace{-1.5ex}
\begin{equation} \label{plow1sub2}
	\underline{p}^{a}=\dfrac{\left(2\Delta_{H}+\Delta_{S}\right)-\alpha\Delta_{S}\left(\Delta_{L}+2\right)}{\left(2\Delta_{H}+\Delta_{S}\right)-\alpha\Delta_{S}\left(\Delta_{H}-2\right)}v 
\end{equation}

Because advertising firms never charge prices in $[\underline{p}^{n},\underline{p}^{a}]$, $F^{a}(p)=0$ for all $p \in[\underline{p}^{n}\text{, }\underline{p}^{a}]$. Setting $F^{a}(p)=0$ in Expression~\eqref{prof0sub2}, and setting $\pi^{n}(p)=\pi^{n}(v)$ we can solve for $F^{n}(p)$ for $p\in[\underline{p}^{n}\text{, }\underline{p}^{a}]$. 

\vspace{-1.5ex}
\begin{equation} \label{F0sub2low}
	F^{n}(p)=1+\dfrac{2\alpha(1-\mu^{2})}{2(1-\alpha)}-\dfrac{1-\alpha\Delta_{L}}{2(1-\alpha)}\left(\dfrac{v}{p}-1\right)
\end{equation}

Setting $F^{n}(\underline{p}^{n})=0$ in Expression~\eqref{F0sub2low} yields  

\vspace{-1.5ex}
\begin{equation} \label{plow0sub2}
	\underline{p}^{n}=\dfrac{1-\alpha\Delta_{L}}{3-\alpha\Delta_{H}}v  
\end{equation} 

Finally, solving $\pi^{n}(v)=\pi^{a}(v)$ for $\alpha$ yields $\alpha=\left(\Delta_{H}v-4A\right)/\left(\Delta_{S}v\right)$. A sufficient condition for $\alpha\in(0\text{, }1)$ is that $A\in\left(\Delta_{L}v/4,\Delta_{H}v/4\right)$.

We can substitute $\alpha$ into Expression~\eqref{F0sub2} through \eqref{plow0sub2} to obtain the equilibrium distribution function and bounds on firm price supports. However, before considering this equilibrium fully characterized, it must be confirmed that $\underline{p}^{a}>\underline{p}^{n}$ and that both conditional distribution functions are increasing and take values between 0 and 1 on the interior of their supports. Whereas this is readily done for $F^{a}$, for $F^{n}$ on $[\underline{p}^{a}\text{, }v]$, it entails the following additional condition (which guarantees that the numerator in the long fraction term in Expression~\eqref{F0sub2} is negative): 

\vspace{-1.5ex}
\begin{equation} \label{condA}
	A(\Delta_{H}-2) > \mu^{2}(2-\mu^{2})\frac{v}{4}> 0   
	\vspace{-1.5ex}
\end{equation}

A necessary condition for this inequality to hold is that $\mu>\sqrt{3}-1 \approx 0.73$, which turns the the left-hand side of the expression positive. Additionally, in order to ensure that Condition~\eqref{condA} holds, it must be the case that $\mu>0.83$ when $A$ is close to $\Delta_{H}v/4$, and $\mu$ must approach 1 as $A$ comes closer to $\Delta_{H}v/12$. That is, $A$ must be large enough that a non-advertising firm does not wish to deviate even though a firm that advertises captures a larger market share than a non-advertising rival that underprices it. 
 
Figure~\ref{eqsp} shows that, Condition~\eqref{condA}, together with the restriction that $A\in\left(\Delta_{L}v/4,\Delta_{H}v/4\right)$, imply that this equilibrium, represented by the dark-gray region at the left of the figure, exists only when $\mu$ is sufficiently high. By plotting profits across the dark-gray region of Figure~\ref{eqsp}, it is possible to show that profits are lower than would result in the baseline throughout this portion of the $\mu\times A$ parameter space.

Now suppose instead that following Expressions~\eqref{F0sub2} and \eqref{F1sub2}, that $\underline{p}^{n}\geq\underline{p}^{a}$. Following the same steps used in Expressions~\eqref{plow1sub2} through \eqref{plow0sub2} with subscripts 0 and 1 reversed, we can again derive $\underline{p}^{n}$ and $\underline{p}^{a}$. Comparison of $\underline{p}^{n}$ and $\underline{p}^{a}$ shows that $\underline{p}^{n}<\underline{p}^{a}$ for all values of $\mu$ that satisfy Condition~\eqref{condA}, a contradiction.        
\vspace{1.5ex}

\noindent \textbf{No other equilibria where ads partially substitute low prices ($\boldsymbol{v=\bar{p}^{a}>\bar{p}^{n}>\underline{p}^{a}}$):} 

An advertising firm's profit for $p\in(\bar{p}^{n}\text{, }v]$ is given by Expression~\eqref{prof1sub} in Section~\ref{Effective Advertising}. In equilibrium, $\pi^{a}(v)=\pi^{a}(p)$ for $p\in (\bar{p}^{n}\text{, }v)$. Solving this equation for $F^{a}(p)$ yields Expression~\eqref{F1sub}. Recall from Section~\ref{Effective Advertising} that Condition~\eqref{cond_alpha_Reversed} is necessary for a firm that does not advertise to not deviate to some $p>\bar{p}^{n}$. Suppose that Condition~\eqref{cond_alpha_Reversed} holds.     

Next, simultaneously solving $\pi^{n}(p)=\pi^{n}(\bar{p}^{n})$ and $\pi^{a}(p)=\pi^{n}(p)$ for $p$ in the overlapping part of firm supports we can solve for $F^{n}(p)$ and $F^{a}(p)$ in terms of $\bar{p}^{n}$. Doing so, yields 

\vspace{-1.5ex}
\begin{equation} \label{F0sub2low_bar_p}
	F^{n}(p)=1+\dfrac{2\Delta_{H}-\Delta_{S}\left[(1+\mu)^{2}-\alpha\left(\Delta_{H}-2\right)\right]}{(1-\alpha)\Delta_{S}(4-\Delta_{S})}\left(\dfrac{\bar{p}^{n}}{p}-1\right) 
	\vspace{-1.5ex}
\end{equation}

Condition~\eqref{cond_alpha_Reversed} implies that $F^{n}(p)>1$, a contradiction.
\vspace{1.5ex}

\noindent \textbf{No equilibria where ads can complement low prices ($\boldsymbol{v=\bar{p}^{n}>\bar{p}^{a}}$):} \vspace{1.5ex}

The expected profit of a non-advertising firm at price $p\in(\bar{p}^{a}\text{, }v]$ is 

\vspace{-1.5ex}
\begin{equation} \label{prof0comp}
	\pi^{n}(p)=\underbrace{\alpha p\frac{(1-\mu)^{2}}{4}}_{\text{rival advertises}}+\underbrace{(1-\alpha)\frac{p}{4}\left[ 1+2(1-F^{n}(p)) \right]}_{\text{rival doesn't advertise}} 
\end{equation}

In equilibrium, $\pi^{n}(v)=\pi^{n}(p)$ for $p\in (\bar{p}^{a}\text{, }v)$. Solving this for $F^{n}(p)$ yields

\vspace{-1.5ex}
\begin{equation} \label{F0comp}
	F^{n}(p)=1-\dfrac{1-\alpha\Delta_{L}}{2(1-\alpha)}\left(\dfrac{v}{p}-1\right) \vspace{-1.5ex}
\end{equation}

In equilibrium, an advertising firm must not wish to deviate to a price above $\bar{p}^{a}$. A firm that does so, earns profit 
\vspace{-1.5ex}
\begin{equation}\label{prof1comp_deviation}
	\pi^{a}(p) = \frac{p}{4}\left\{ \alpha +(1-\alpha)[(1+\mu)^{2}+2(1-\mu^{2})(1-F^{n}(p))] \right\}-A \vspace{-1.5ex}
\end{equation}

Substituting the value of $F^{n}(p)$ calculated in Expression~\eqref{F0comp}, we find that $\mathrm{d}\pi^{a}(p)/\mathrm{d}p>0$---that is, the advertising firm would always wish to deviate to a higher $\bar{p}^{a}$ for any $\bar{p}^{a}<v$.
\vspace{-1.5ex}

\noindent \textbf{No equilibria with gaps in the support ($\boldsymbol{\bar{p}^{n}<\underline{p}^{a}}$):} \vspace{1.5ex}

Suppose that $\bar{p}^{n}<\underline{p}^{a}$. Then $\pi^{n}(\bar{p}^{n})=\bar{p}^{n}\left\{ \alpha \left[(1-\mu)^{2}+2(1-\mu^{2}) \right]+(1-\alpha)\right\}/4$, which is increasing in $\bar{p}^{n}$ up to $\underline{p}^{a}$, a contradiction. \qedhere
\end{proof}




%%%%%%%%%%     SUPPLEMENTAL APPENDIX     %%%%%%%
\section*{Supplemental Online Appendix}


%%%%   DERIVATION OF MIXED ADVERTISING OLIGOPOLY EQUILIBRIUM   %%%%%%%%%
\subsection*{Derivation of Mixed Advertising Strategy Oligopoly Equilibrium}

Here, we use Expressions~\eqref{pi0} and \eqref{pi1} to derive the mixed advertising strategy oligopoly equilibrium discussed in Section~\ref{Oligopoly}.

Setting $\pi^{a}(v)=\pi^{a}(p)$ for $p\in [p^{*},v]$, we can use Expression~\eqref{pi1} to solve for $F^{a}(p)$ in terms of $\alpha$.

\vspace{-1.5ex}
\begin{equation}\label{F1}
	\begin{array}{l}
		\displaystyle{F^{a}(p)=1-\left(\frac{v}{p}-1 \right)} \\
		\times\displaystyle{\left[\frac{(1-\alpha)^{N-1}[\gamma(1)]^{2}+\sum\limits_{k=1}^{N-2}\binom{N-1}{k}\alpha^{k}(1-\alpha)^{N-1-k}[\gamma(k+1)]^{2}+\alpha^{N-1}}{\sum\limits_{k=1}^{N-2}\binom{N-1}{k}\alpha^{k}(1-\alpha)^{N-1-k}2k[\gamma(k+1)]^{2}+2(N-1)\alpha^{N-1}}\right]}
	\end{array} 
	\vspace{-1.5ex}
\end{equation}

Setting $F^{a}(p^{*})=0$ yields $p^{*}$ in terms of $\alpha$. 

\vspace{-1.5ex}
\begin{equation*}
	p^{*}=\frac{(1-\alpha)^{N-1}[\gamma(1)]^{2}+\sum\limits_{k=1}^{N-2}\binom{N-1}{k}\alpha^{k}(1-\alpha)^{N-1-k}[\gamma(k+1)]^{2}+\alpha^{N-1}}{(1-\alpha)^{N-1}[\gamma(1)]^{2}+\sum\limits_{k=1}^{N-2}\binom{N-1}{k}\alpha^{k}(1-\alpha)^{N-1-k}[\gamma(k+1)]^{2}(1+2k)+\alpha^{N-1}(2N-1)} v \vspace{-1.5ex}
\end{equation*}

Similarly, setting $\pi^{n}(p^{*})=\pi^{n}(p)$ for $p\in [\underline{p},p^{*}]$, we can use Expression~\eqref{pi0} to solve for $F^{n}(p)$ in terms of $\alpha$ and $p^{*}$ as follows. 

\vspace{-1.5ex}
\begin{equation}\label{F0}
	\begin{array}{l}
		\displaystyle{F^{n}(p)=1-\left(\frac{p^{*}}{p}-1 \right)} \\
		\times\displaystyle{\left[
\frac{(1-\alpha)^{N-1} + \sum\limits_{k=1}^{N-1}\binom{N-1}{k}\alpha^{k}(1-\alpha)^{N-1-k}\left[(1-\mu)^{2}+2k(1-\mu)\gamma(k)\right]}
{2(N-1)(1-\alpha)^{N-1}+\sum\limits_{k=1}^{N-2}\binom{N-1}{k}\alpha^{k}(1-\alpha)^{N-1-k}2(N-1-k)(1-\mu)^{2}}\right]}
	\end{array} 
	\vspace{-1.5ex}
\end{equation}

Setting $F^{n}(\underline{p})=0$ yields $\underline{p}$ in terms of $p^{*}$ and $\alpha$.\footnote{Note that we use the fact that $2[N-1-(N-1)](1-\mu)^2/N^2=0$, which says that when all other firms are advertising it is impossible for consumers to observe two firms that do not advertise.} 

\vspace{-1.5ex}
\begin{equation*}
	\underline{p}=\frac{(1-\alpha)^{N-1} + \sum\limits_{k=1}^{N-1}\binom{N-1}{k}\alpha^{k}(1-\alpha)^{N-1-k}\left[(1-\mu)^{2}+2k(1-\mu)\gamma(k)\right]}
{(1-\alpha)^{N-1} (2N-1)+ \sum\limits_{k=1}^{N-1}\binom{N-1}{k}\alpha^{k}(1-\alpha)^{N-1-k}\left\{(1-\mu)^{2}\left[2(N-k)-1\right]+2k(1-\mu)\gamma(k)\right\}} p^{*}
	\vspace{-1.5ex}
\end{equation*}

At $p^{*}$, firms must be indifferent between advertising and not advertising. Setting $\pi^{n}(p^{*})=\pi^{a}(p^{*})$, we can implicitly solve for $\alpha$ in terms of $A$, $\mu$ and $N$. 

\vspace{-1.5ex}
\begin{equation}\label{alphasub}
	\begin{array}{l}
		\displaystyle{\frac{N^{2}A}{p^{*}}=\alpha^{N-1}\left\{1-(1-\mu)^{2}+2(N-1)\left[1-(1-\mu)\gamma(N-1)\right]\right\}} \\
		\displaystyle{+(1-\alpha)^{N-1}\left[[\gamma(1)]^{2}-1\right]+\sum\limits_{k=1}^{N-2}\binom{N-1}{k}\alpha^{k}(1-\alpha)^{N-1-k}} \\
		\displaystyle{\times\left\{[\gamma(k+1)]^{2}-(1-\mu)^{2}+2k\left[ [\gamma(k+1)]^{2}-\gamma(k)(1-\mu)\right]\right\}}
	\end{array} 
\end{equation}


\subsubsection*{Checking for deviations} 

We need to check for two types of deviations: (1) price deviations and (2) advertising deviations. We start by showing when firms have no incentives to deviate from the pricing strategy described above. There are two types of deviations that we must consider. First, when a firm advertises, it must not wish to deviate to a price below $p^{*}$.  Second, when a firm does not advertise, it must not wish to deviate to a price above $p^{*}$. \vspace{1.5ex}

\noindent \textbf{When a firm advertises, it does not want to lower prices below $\boldsymbol{p^{*}}$:} By deviating to a price $p<p^{*}$, the firm expects deviation profits $\pi^{a}_{d}(p)$: 

\vspace{-1.5ex}
\begin{equation}\label{pi1dev}
	\begin{array}{l}
		\displaystyle{\pi^{a}_{d}(p)=\frac{p}{N^{2}}\Bigg\{ (1-\alpha)^{N-1}\left[[\gamma(1)]^{2}+2(N-1)(1-\mu)\gamma(1)\left(1-F^{n}(p)\right)\right]} \\
		\displaystyle{+\alpha^{N-1}\left(2N-1\right)+\sum\limits_{k=1}^{N-2}\binom{N-1}{k}\alpha^{k}(1-\alpha)^{N-1-k}} \\
		\displaystyle{\times\left[[\gamma(k+1)]^{2}\left(1+2k\right)+2(N-k-1)(1-\mu)\gamma(k+1)\left(1-F^{n}(p)\right)\right]\Bigg\}-A}  
	\end{array} 
	\vspace{-1.5ex}
\end{equation}

There are no incentives to lower prices to some $p'<p^{*}$ if expected deviation profits are lower than the expected equilibrium profit, that is, if $\pi^{a}(p^{*})>\pi^{a}_{d}(p')$. This is the case whenever the derivative of expected deviation profits with respect to $p$ is positive, so that lowering prices decreases expected profits. Substituting $F^{n}(p)$ from Expression~\eqref{F0} into Expression~\eqref{pi1dev}, yields the following expression: 

\vspace{-1.5ex}
\begin{equation}\label{pi1dev2}
	\begin{array}{l}
		\displaystyle{\pi^{a}_{d}(p)= \frac{p}{N^{2}}\left[\Psi_{1}+\Phi_{1} \Gamma_{1}\left(\frac{p^{*}}{p}-1\right)\right]} \\
		\displaystyle{\text{where}} \\
		\displaystyle{\Psi_{1}\equiv (1-\alpha)^{N-1}[\gamma(1)]^{2}+\alpha^{N-1}(2N-1)} \\
		\displaystyle{\quad\ \ +\sum\limits_{k=1}^{N-2}\binom{N-1}{k}\alpha^{k}(1-\alpha)^{N-1-k}[\gamma(k+1)]^{2}(1+2k)} \\
		\displaystyle{\Phi_{1}\equiv (1-\alpha)^{N-1}2(N-1)(1-\mu)\gamma(1)} \\
		\displaystyle{+\sum\limits_{k=1}^{N-2}\binom{N-1}{k}\alpha^{k}(1-\alpha)^{N-1-k}2(N-k-1)(1-\mu)\gamma(k+1)}  \\
		\displaystyle{\Gamma_{1}\equiv \frac{(1-\alpha)^{N-1} + \sum\limits_{k=1}^{N-1}\binom{N-1}{k}\alpha^{k}(1-\alpha)^{N-1-k}\left((1-\mu)^{2}+2k(1-\mu)\gamma(k)\right)}{(1-\alpha)^{N-1}2(N-1)+\sum\limits_{k=1}^{N-2}\binom{N-1}{k}\alpha^{k}(1-\alpha)^{N-1-k}2(N-1-k)(1-\mu)^{2}}}
	\end{array} 
	\vspace{-1.5ex}
\end{equation}

Taking the derivative of Expression~\eqref{pi1dev2} with respect to $p$ yields the following expression, which is independent of $p$. 

\begin{equation}\label{Dpi1dp}
	\frac{d\pi^{a}_{d}}{dp}=\frac{\Psi_{1}-\Phi_{1}*\Gamma_{1}}{N^{2}}
\end{equation} 

As long as Expression~\eqref{Dpi1dp} is positive, the firm does not wish to decrease its price below $p^{*}$. \vspace{1.5ex}

\noindent \textbf{When a firm does not advertise, it does not want to increase prices above $\boldsymbol{p^{*}}$:} By deviating to a price $p>p^{*}$, the firm expects deviation profits $\pi^{n}_{d}(p)$: 

\vspace{-1.5ex}
\begin{equation}\label{pi0dev}
	\begin{array}{l}
		\displaystyle{\pi^{n}_{d}(p)= \frac{p}{N^{2}}\Bigg\{(1-\alpha)^{N-1}\!+\alpha^{N-1}\left[(1-\mu)^{2}\!+2(N-1)(1-\mu)\gamma(N-1)\left( 1-F^{a}(p)\right)\right]} \\
		\displaystyle{+\sum\limits_{k=1}^{N-2}\binom{N-1}{k}\alpha^{k}(1-\alpha)^{N-1-k}\left[(1-\mu)^{2}+2k(1-\mu)\gamma(k)\left( 1-F^{a}(p)\right)\right]\Bigg\}}
	\end{array} \vspace{-1.5ex}
\end{equation}

The firm does not have an incentive to increase prices to some $p'>p^{*}$ when the derivative of expected deviation profits with respect to $p$ is negative, such that raising prices decreases expected profits. Substituting $F^{a}(p)$ from Expression~\eqref{F1} into Expression~\eqref{pi0dev}, yields the following expression: 

\vspace{-1.5ex}
\begin{equation}\label{pi0dev2}
	\begin{array}{l}
		\displaystyle{\pi^{n}_{d}(p)=  \frac{p}{N^{2}}\left[\Psi_{0}+\Phi_{0} \Gamma_{0}\left(\frac{v}{p}-1\right)\right]} \\
		\displaystyle{\text{where}} \\
		\displaystyle{\Psi_{0}= (1-\alpha)^{N-1}+ \sum\limits_{k=1}^{N-2}\binom{N-1}{k}\alpha^{k}(1-\alpha)^{N-1-k}(1-\mu)^{2} +\alpha^{N-1}(1-\mu)^{2}} \\
		\displaystyle{\Phi_{0}= \sum\limits_{k=1}^{N-2}\binom{N-1}{k}\alpha^{k}(1-\alpha)^{N-1-k}2k(1-\mu)\gamma(k)+\alpha^{N-1}2(N-1)(1-\mu)\gamma(N-1)}  \\
		\displaystyle{\Gamma_{0}=\frac{(1-\alpha)^{N-1}[\gamma(1)]^{2}+\sum\limits_{k=1}^{N-2}\binom{N-1}{k}\alpha^{k}(1-\alpha)^{N-1-k}[\gamma(k+1)]^{2}+\alpha^{N-1}}{\sum\limits_{k=1}^{N-2}\binom{N-1}{k}\alpha^{k}(1-\alpha)^{N-1-k}2k[\gamma(k+1)]^{2}+2(N-1)\alpha^{N-1}}}
	\end{array} \vspace{-1.5ex}
\end{equation}

Taking the derivative of Expression~\eqref{pi0dev2} with respect to $p$ yields the following expression, which is independent of $p$. 

\begin{equation}\label{Dpi0dp}
	\frac{d\pi^{n}_{d}}{dp}=\frac{\Psi_{0}-\Phi_{0}*\Gamma_{0}}{N^{2}}
\end{equation} 

As long as Expression~\eqref{Dpi0dp} is negative, the firm does not wish to increase prices above $p^{*}$.


\subsubsection*{Checking for advertising deviations}

We next need to confirm that firms do not want to alter their advertising strategy. In particular, to place bounds on the set of mixed advertising strategy equilibria, we construct no-deviation conditions from pure advertising strategy equilibria ($\alpha=0$ and $\alpha=1$). Following the procedure used to derive Proposition~\ref{PureAd} we can show that when firms either advertise with certainty or not at all, the equilibrium price distribution equals $F^b(p)=1-(v/p-1)/(2N-2)$, which has lower bound $\underline{p}^b=v/(2N-1)$. Equilibrium expected profits equal $\pi^b=v/N^{2}$ when $\alpha=0$ and $\pi^b-A$ when $\alpha=1$.  \vspace{1.5ex}

\noindent \textbf{Deviation from $\boldsymbol{\alpha=1}$:} Because equilibrium expected profits in this case equal to $v/N^{2}-A$, it must be that $A<v/N^{2}$. Moreover, by deviating to not advertising, a firm obtains deviation profit 

\begin{equation}\label{devpiadv}
	\pi^{d}(p)=\frac{p}{N^{2}}\left[(1-\mu)^{2}+2(N-1)(1-\mu)\gamma(N-1)\left(1-F^b(p)\right)\right] 
\end{equation}

When all of its rivals advertise, a firm that deviates to not advertising ceases to be prominent to all prominent rivals and ends up with a lower fraction of captives $(1-\mu)^{2}/N^{2}$ and competes for $2(N-1)(1-\mu)\gamma(N-1)/N^{2}$ shoppers while avoiding the cost of advertising, $A$. The firm does best by deviating to no advertising when it is charging $\underline{p}$, in which case it attracts all shoppers.\footnote{Subtracting pure strategy profit $\pi^b$ from Expression~\eqref{devpiadv}, the derivative of the gains from deviating with respect to $p$ is: \vspace{-1.5ex} $$\frac{d(\pi_{d}(p)-\pi^b(p))}{dp}=\frac{(1-\mu)^{2}-1+[2(1-\mu)(N-1+\mu)-2(N-1)](1-F^b(p)-p[F^{b}]'(p))}{N^{2}}=\frac{-\mu(1-\mu)}{N(N-1)}<0\vspace{-1.5ex}$$ where the second equality follows by substituting $F^b(p)$.}

Firms do not wish to deviate if $\pi(\underline{p})$ is greater than $\pi^{d}(\underline{p})$.  That is, if 

\vspace{-1.5ex}
\begin{equation}
	\frac{\underline{p}}{N^{2}}\left(2N-1\right)-A>\frac{\underline{p}}{N^{2}}\left[(1-\mu)^{2}+2(N-1)(1-\mu)\gamma(N-1)\right] \vspace{-1.5ex}
\end{equation}

which reduces to,

\begin{equation}\label{Alow}
	A<\frac{\mu[2(N-1)+\mu]}{(2N-1)N^{2}}v\equiv A_{L} \vspace{1.5ex}
\end{equation}

\noindent \textbf{Deviation from $\boldsymbol{\alpha=0}$:} By deviating to advertising, a firm becomes prominent and earns deviation profit 

\vspace{-2.5ex}
\begin{equation}\label{devpina}
	\pi^{d}(p)=\frac{p}{N^{2}}\left[[\gamma(1)]^{2}+2(N-1)(1-\mu)\gamma(1)\left(1-F(p)\right)\right]-A \vspace{-2.5ex}
\end{equation}

In this case, the firm does best by deviating to advertising when it is charging $v$ and, therefore, only sells to its captives.\footnote{Subtracting $\pi^b$ from Expression~\eqref{devpina}, the derivative of the gains from deviating with respect to $p$ is: \vspace{-1.5ex}
$$\frac{d(\pi^{d}(p)-\pi^b(p))}{dp}=\frac{[\gamma(1)]^{2}-1+\left[2(N-1)(1-\mu)\gamma(1)-2(N-1)\right](1-F^b(p)-p[F^{b}]'(p))}{N^{2}}=\frac{\mu\left[1+(N-1)\mu\right]}{N}>0\vspace{-1.5ex}$$ where the second equality follows by substituting $F^b(p)$.} Thus, the no-deviation condition is 

\vspace{-1.5ex}
\begin{equation}
	\frac{v}{N^{2}}>\frac{v}{N^{2}}[\gamma(1)]^{2}-A \vspace{-1.5ex}
\end{equation}

which can be rewritten as

\begin{equation}\label{Ahigh}
	A>\frac{\left[ [\gamma(1)]^{2}-1\right]}{N^{2}}v\equiv A_{H}
\end{equation}

Therefore, whenever $A\in(A_{L},A_{H})$, firms play mixed advertising strategies. 



%%%%%%%   SEQUENTIAL DUOPOLY   %%%%%%%%%%%%
\subsection{Sequential Duopoly}\label{Sequential Duopoly}

In the model of Section~\ref{Duopoly}, we supposed that firms simultaneously set their advertising and pricing strategies. However, if we view advertising decisions as ones that require more long term planning than pricing, it may be more appropriate to suppose that firms first simultaneously decide whether or not to advertise, then after observing the advertisement decisions they choose prices, following which consumers form consideration sets and make purchasing decisions. As we discuss in this section, the main results from Section~\ref{Duopoly} persist, though market segmentation is no longer absolute: that is, instead of exclusively focusing on different consumer segments, when one firm advertises and the other does not, the advertising firm focuses more on its captives and less on shoppers, whereas the other focuses on its diminished number of captives and more on shoppers than its advertising rival.

More precisely, in this sequential advertising and pricing game, there are three alternative advertising outcomes in the first stage: both firms advertise, neither does so, or one firm advertises whereas the other does not. In subgames where both firms advertise or neither does so, the equilibrium outcome is the same as in the pure strategy cases in Theorem~\ref{all_equilibria}. In contrast, suppose that firm $i$ advertises, whereas firm $j$ does not. In this case, if we suppose that price dispersion continues to persist, with the advertising and non-advertising firms drawing prices from respective continuous distribution functions $F^{a}(p)$ and $F^{n}(p)$ (not necessarily the ones derived in Section~\ref{Effective Advertising}), then the profit functions of the advertising and non-advertising firms may be written as, respectively, 

\vspace{-1.5ex}
\begin{equation} \label{pia} 
	\pi^{a}(p)=\frac{p}{4}\left[(1+\mu)^{2} +2(1-\mu^{2})\left(1-F^{n}(p)\right)\right]-A\text{,} \vspace{-1.5ex}
\end{equation}

\begin{equation} \label{pin}
	\pi^{n}(p)=\frac{p}{4}\left[(1-\mu)^{2}+2(1-\mu^{2})\left(1-F^{a}(p)\right)\right] 
\end{equation}

Equations~\eqref{pia} and \eqref{pin} rule out the equilibrium studied in Section~\ref{Effective Advertising}, in which an advertising firm always sets a higher price than its non-advertising rival. Suppose to the contrary that $F^{n}(p)$ and $F^{a}(p)$ had respective supports $[\underline{p}^{n}\text{, }\underline{p}^{a}]$ and $[\underline{p}^{a}\text{, }v]$. Then Equations~\eqref{pia} and \eqref{pin} reduce to, respectively, $\pi^{a}(p)=(1+\mu)^{2}p/4 -A$ and $\pi^{n}(p)=[(1-\mu)^{2} + 2(1-\mu^{2})]p/4$. But in this case, both firms would wish to deviate to $v$, a contradiction. 

\begin{lemma} \label{sequential_support}
In the equilibrium of the subgame of the sequential advertising and pricing game in which one firm advertises and the other does not, the supports of firm pricing distributions are the same, do not have any breaks, and are bounded from above by $v$. In equilibrium, one firm has an atom at $v$.
\end{lemma}

For concision, we state Lemma~\ref{sequential_support} without proof, referring the reader, instead, to Narasimhan (1988) and Astorne-Figari and Yankelevich (2014), which derive similar results in asymmetric extensions of Varian's (1980) model of sales.\footnote{Roughly, firms make no profit above $v$ and, with its advertising action fixed, a firm pricing below its rival's lower bound gets the same number of captives and all shoppers at any such price and stands to profit by shifting its price upward. Firms cannot have an atom at the same price by a similar atom undercutting argument used in Lemma~\ref{support_advertising}. Firms cannot have breaks in their supports because their rivals expect to capture the same number of shoppers everywhere along the break, implying that profits are increasing in price along the break, a contradiction of mixed strategy equilibrium. Finally, convex supports imply that firms cannot have an atom anywhere below $v$ because otherwise a firm can profitably shift mass above the atom slightly below it to capture shoppers with strictly higher probability.} With these restrictions on firm pricing, it becomes straightforward to derive firms' equilibrium price distributions and profits in the subgame in which only one firm advertises by following a similar approach to that used in Sections~\ref{Baseline} and \ref{Effective Advertising}.

\begin{proposition} \label{Ad_Sequential}
In the subgame perfect Nash equilibrium outcome of the game of advertising, then pricing, when $A\in\left(\left[\mu(3+\mu)-2\right]\mu v/[4(3-\mu)],\mu(2+\mu)v/4\right)$, only one firm advertises in equilibrium. Firms price according to 

\vspace{-1.5ex}
\begin{equation*}
	F^{n}(p)=\left[1+\frac{1+\mu}{2(1-\mu)}\right]\left(1-\frac{\underline{p}}{p}\right)\text{,}\qquad F^{a}(p)=\left[1+\frac{1-\mu}{2(1+\mu)}\right]\left(1-\frac{\underline{p}}{p}\right) \vspace{-1.5ex}
\end{equation*}

over support $[\underline{p}\text{, }v]$, where $\underline{p}=(1+\mu)v/(3-\mu)$. In equilibrium, the advertising firm has an atom at $v$ and $F^{b}(p)>F^{n}(p)>F^{a}(p)$. When only one firm advertises, it always earns higher profit than in the baseline, whereas its rival earns higher profit for $\mu\in(0\text{, }0.56)$. 
\end{proposition}

\begin{proof}
Working backward, suppose that only one firm advertises in equilibrium. From Lemma~\ref{sequential_support}, we know that it must be that $\pi^{n}(\underline{p})=\pi^{n}(p)$ for all $p\in [\underline{p},v)$. Using Expression~\eqref{pin}, we obtain $F^{a}(p)$ in the statement of Proposition~\ref{Ad_Sequential}. Likewise, $\pi^{a}(\underline{p})=\pi^{a}(p)$ for all $p\in [\underline{p},v)$, so that using Expression~\eqref{pin}, we obtain $F^{n}(p)$ in the statement of Proposition~\ref{Ad_Sequential}.

Note that both $F^{n}(p)$ and $F^{a}(p)$ equal to zero at $\underline{p}$. Moreover, setting $F^{n}(p)=1$ and solving for $\underline{p}$ yields $\underline{p}=(1+\mu)v/(3-\mu)$. Substituting this expression into $F^{a}(v)$ yields $F^{a}(v)=(3+\mu)(1-\mu)/\left[(3-\mu)(1+\mu)\right]\in(0,1)$, which implies that the advertising firm has an atom at $v$.   

Comparing $F^{n}(p)$ and $F^{a}(p)$ yields $F^{n}(p)>F^{a}(p)$ if and only if $(1+\mu)^2>(1-\mu)^2$, which holds for all $\mu\in(0\text{, }1)$. Further, substituting the expression for $\underline{p}$ into $F^{n}(p)$ and comparing to Equation~\eqref{distribution} yields $F^{b}(p)>F^{n}(p)$, where $F^{b}(p)$ is the price distribution following the subgames in which both firms advertise in the first stage or neither firm advertises in the first stage.

To see that it is a subgame perfect outcome for one firm to advertise whenever $A\in\left(\left[\mu(3+\mu)-2\right]\mu v/[4(3-\mu)],\mu(2+\mu)v/4\right)$, we must show that in this outcome, the advertising firm earns more profit than when neither firm advertises and that the non-advertising firm earns more profit than when both firms advertise. This requires that $\pi^{a}>v/4$ and $\pi^{n}>v/4-A$ (expected profits in the pure advertising strategy equilibrium with $\alpha=0$, $v/4$, and the pure advertising strategy equilibrium with $\alpha=1$, $v/4-A$, from Theorem~\ref{all_equilibria}. The respective comparisons lead to the specified range for $A$, which is illustrated in Figure~\ref{eqsp_sequential} for $\mu\in(0\text{, }1)$. From this, it immediately follows that in this range, the advertising firm always earns higher profit than if its rival advertised or if neither firm advertised. It also follows that in this range, the non-advertising firm earns higher profit than if both firms advertised, but not necessarily if neither firm advertised.

Setting $\pi^{n}>v/4$, it follows that $\pi^{n}$ is greater whenever $\mu\in(0\text{, }0.56)$. That is, for sufficiently low $\mu$, both firms earn higher profits than in the baseline. 
\qedhere
\end{proof}

Proposition~\ref{Ad_Sequential} implies that unlike in the simultaneous pricing and advertising game, when one firm advertises and the other does not in equilibrium, both price over the same support. However, although the market is not perfectly segmented as in Proposition~\ref{Ad_Subs_Low_Price}, because $F^{a}$ dominates $F^{n}$ in the first order stochastic sense, as in Section~\ref{Effective Advertising}, the advertising firm focuses more on catering to its larger number of captives, whereas its non-advertising rival prices more aggressively to increase its probability of capturing all shoppers.

\begin{figure}[!thb]
\center
  \includegraphics[width=0.85\textwidth]{eqspace_sequential.eps}
  \caption{Equilibrium over parameter space $\mu\times A$}
  \label{eqsp_sequential}
\end{figure}
    
Figure~\ref{eqsp_sequential} captures the parameter space over which only one firm advertises in the subgame perfect outcome. As in Section~\ref{Duopoly}, this happens for an intermediate range of $A$. When $A>\mu(2+\mu)v/4$, neither firm advertises, and when $A<\left[\mu(3+\mu)-2\right]\mu v/[4(3-\mu)]$, both do so in equilibrium. In each case, firms price according to the baseline distribution $F^{b}$ from Section~\ref{Baseline} and both firms expect to set lower prices than when only one firm advertises. Intuitively, this occurs for the same reasons that firms set higher prices in Section~\ref{Effective Advertising}: advertising diminishes the fraction of shoppers and allows firms to focus on different market segments. 

In the light-gray region in Figure~\ref{eqsp_sequential}, the advertising firm always earns higher profits than it would without advertising, while the non-advertising firm earns higher profits than it would if both firms advertised (otherwise they would deviate in the first stage). Moreover, if $\mu$ is not too high (below approximately 0.56), the non-advertising firm earns higher profit than if neither firm advertised, meaning that as in Section~\ref{Effective Advertising}, advertising for attention has the potential to raise aggregate and individual firm profits. 



%%%%%%%%%   ASYMMETRIC ADVERTISING COSTS   %%%%%%%%%%%%
\subsection{Asymmetric Advertising Costs} \label{Asymmetric Costs} 

Returning to the simultaneous pricing and advertising game in Section~\ref{Duopoly}, without loss of generality, let us instead suppose that firm 1 faces advertising cost $A_1=\sigma A$ for $\sigma\in[0\text{, }1]$ and firm 2 faces advertising cost $A_2=A$. Rather than undertaking an exhaustive analysis as we did in Section~\ref{Duopoly}, we are interested in whether and when equilibria similar to those characterized in Sections~\ref{Effective Advertising} or \ref{Sequential Duopoly} persist. 

Assume first that $\sigma<1$, so that $A_1<A_2$. Suppose again, that $F^{n}_{i}(p)$ and $F^{a}_{i}(p)$ have respective supports $[\underline{p}^{n}\text{, }\underline{p}^{a}]$ and $[\underline{p}^{a}\text{, }v]$, where subscript $i$ again denotes firm $i$. From Expressions~\eqref{F1sub} and \eqref{F0sub}, we know that $F^{a}_{i}(p)$ and $F^{n}_{i}(p)$ are functions of $\alpha_i$, which is endogenously determined by the cost of advertising. Using the procedure in Section~\ref{Effective Advertising}, $F^{a}_{i}(p)$ and $F^{n}_{i}(p)$, it then follows that the price distribution of one firm contains an atom at $\underline{p}^{a}$ (at the top of conditional distribution $F^{n}_{i}(p)$) and the price distribution of the other at $v$.\footnote{Specifically, under our assumptions on $A_1$ and $A_2$ it can be shown that $F^{n}_{2}(p)$ contains the atom at $\underline{p}^{a}$ and $F^{a}_{1}(p)$ contains the atom at $v$. Computations are available upon request from the authors.} However, if one firm has an atom at $\underline{p}^{a}$, an advertising rival can profit by undercutting the rival, suggesting that complete market segmentation does not occur in equilibrium when advertising costs differ.\footnote{Moreover, an alternative market segmentation equilibrium in which there is a gap between the upper-bound of the supports of $F^{n}_{i}(p)$ and the lower-bound of the supports of $F^{a}_{i}(p)$ cannot exist because the firm with the atom at $F^{n}_{i}(p)$ can profitably shift it slightly upward without losing customers.} 

Consider, instead, a variant of the equilibrium outcome of Proposition~\ref{Ad_Sequential}, in which firm 1 advertises, whereas firm 2 does not.\footnote{Note that unlike in this section, that outcome was possible from firms playing symmetric strategies.} Although markets are not perfectly segmented in this equilibrium, it nevertheless represents a potentially profitable outcome to a counterfactual without advertising. Because the distribution functions in Proposition~\ref{Ad_Sequential} are not functions of $A$, it follows that were firm 1 to advertise in equilibrium---while its rival does not---the distribution functions would be precisely those from Proposition~\ref{Ad_Sequential}. However, because we are now considering a simultaneous pricing and advertising game, we must adjust the conditions for a firm to deviate. As opposed to Section~\ref{Sequential Duopoly}---where a firm deviating from the asymmetric advertising outcome brings about the symmetric price distribution from Expression~\eqref{distribution}---when firms make advertising and pricing decisions simultaneously, the advertising decision of a deviating firm is not observed by its rival before prices are determined.

\begin{proposition} \label{Ad_Asymmetric}
Suppose that $(2-\mu)(1+\mu)\mu/\sigma>\mu(6+\mu+4\mu^2+\mu^3)/(1+\mu)$. Then, there exists an asymmetric equilibrium in which firm 1 advertises, firm 2 does not, and firms price according to

\vspace{-1.5ex}
\begin{equation*}
	F^{n}_{2}(p)=\left[1+\frac{1+\mu}{2(1-\mu)}\right]\left(1-\frac{\underline{p}}{p}\right)\text{,}\qquad F^{a}_{1}(p)=\left[1+\frac{1-\mu}{2(1+\mu)}\right]\left(1-\frac{\underline{p}}{p}\right) \vspace{-1.5ex}
\end{equation*}

over support $[\underline{p}\text{, }v]$, where $\underline{p}=(1+\mu)v/(3-\mu)$. 
\end{proposition}

\begin{proof}
The derivations of $F^{a}_{1}(p)$ and $F^{n}_{2}(p)$ proceed exactly following the same steps used to derive, respectively, $F^{a}(p)$ and $F^{n}(p)$ in the proof of Proposition~\ref{Ad_Sequential}. Instead, here, we focus on firms' no-deviation conditions.

By deviating to not advertising, instead of earning $(1+\mu)^{2}v/4-\sigma A$, firm 1 earns $p[1/4+(1/2)(1-F^{n}_{2}(p))]$, which is decreasing in $p$ starting at $\underline{p}$. Therefore, the no-deviation condition is 

\begin{equation}
	\frac{(2-\mu)(1+\mu)\mu v}{4\sigma(3-\mu)}>A  
\end{equation} 

Similarly, by deviating to advertising, instead of earning $(1-\mu^2)(3+\mu)v/[4(3-\mu)]$, firm 2 earns $[3+(10-\mu)\mu]v/[4(3-\mu)(\mu+1)]$, which is increasing in $p$ up to $v$. Note that in computing firm 2's deviation profit, we assume that it slightly undercuts firm 1's atom at $v$. The no-deviation condition is 

\vspace{-1.5ex}
\begin{equation}
	A>\frac{\mu(6+\mu+4\mu^2+\mu^3)v}{4(3-\mu)(1+\mu)} \vspace{-1.5ex}
\end{equation} 

Therefore, unless $(2-\mu)(1+\mu)\mu/\sigma>\mu(6+\mu+4\mu^2+\mu^3)/(1+\mu)$ either firm 1 or firm 2 will wish to deviate. 
\end{proof}

Notice that this is almost precisely the equilibrium advertising and pricing outcome from Proposition~\ref{Ad_Sequential}. Although firms do not perfectly segment the market, firm 1 nevertheless caters primarily to captives, whereas firm 2 caters to shoppers by pricing more aggressively. What is notable about this equilibrium is that firm 1 is not passing down savings from a lower cost of advertising, instead taking advantage of its increased hold on consumer attention by pricing less aggressively than firm 2 or than either firm would in an equilibrium where neither firm advertises.

However, because $\mu(6+\mu+4\mu^2+\mu^3)/(1+\mu)>(2-\mu)(1+\mu)\mu$ for all $\mu\in(0\text{, }1)$, the equilibrium in Proposition~\ref{Ad_Asymmetric} does not exist unless $\sigma$ is sufficiently low.\footnote{We derive firms' no-deviation conditions in the Supplemental Appendix.} This also implies that an equilibrium in which only firm 2 advertises does not exist, and moreover, that this asymmetric equilibrium does not exist when firms face the same advertising cost. 




%%%%%%%%    BIBLIOGRAPHY    %%%%%%%%

\begin{thebibliography}{99}

\bibitem{}Allenby, G.M, Ginter, J.L. (1995).  ``The Effects of In-Store Displays and Feature Advertising on Consideration Sets.''  \textit{International Journal of Research in Marketing} 12, 67-80.

\bibitem{}Anderson, S.P., Baik, A., Larson, N. (2015). ``Personalized Pricing and Advertising: An Asymmetric Equilibrium Analysis.'' \textit{Games and Economic Behavior} 92, 53-73.

\bibitem{}Anderson, S.P., de Palma, A. (2009).  ``Information Congestion.''  \textit{The RAND Journal of Economics} 40, 688-709.

\bibitem{}Anderson, S.P., de Palma, A. (2012).  ``Competition for Attention in the Information (Overload) Age.''  \textit{The RAND Journal of Economics} 43, 1-25.

\bibitem{}Armstrong, M., Zhou, J. (2011). ``Paying for Prominence.'' \textit{The Economic Journal} 121, 368-395.

\bibitem{}Arnold, M., Li, C., Saliba, C., Zhang, L. (2011). ``Asymmetric Market Shares, Advertising and Pricing: Equilibrium with an Information Gatekeeper.'' \textit{The Journal of Industrial Economics} 59, 63-84.

\bibitem{}Arnold, M.A., Zhang, L. (2014). ``The Unique Equilibrium in a Model of Sales with Costly Advertising.'' \textit{Economics Letters} 124, 457-460.

\bibitem{}Astorne-Figari, C., Yankelevich, A. (2014). ``Consumer Search with Asymmetric Price Sampling.'' \textit{Economics Letters} 122, 331-333.

\bibitem{}Bagwell, K., (2007). ``The Economic Analysis of Advertising.''�� \textit{Handbook of Industrial Organization} 3, 1703-1844.

\bibitem{}Baye, M.R., Morgan, J. (2001). ``Information Gatekeepers on the Internet and the Competitiveness of Homogeneous Product Markets.'' \textit{The American Economic Review} 91, 454-474.

\bibitem{}Becker, G., Murphy, K. (1993).  ``A Simple Theory of Advertising as a Good or Bad.'' \textit{The Quarterly Journal of Economics} 108, 941-964.

\bibitem{}Brandenburger, A.M., Nalebuff, B.J. (1996). ``Co-opetition.'' Bantam Doubleday Dell Publishing. New York. 

\bibitem{}Burdett, K., Judd, K. (1983).  ``Equilibrium Price Dispersion.'' \textit{Econometrica} 51, 955-970.

\bibitem{}Butters, G. (1977).  ``Equilibrium Distribution of Sales and Advertising Prices.'' \textit{Review of Economic Studies} 44, 465-492. 

\bibitem{}Chioveanu, I. (2008). ``Advertising, Brand Loyalty and Pricing.'' \textit{Games and Economic Behavior} 64, 68-80.

\bibitem{}Dixit, A. Norman, V. (1978). ``Advertising and Welfare.''\textit{The Bell Journal of Economics} 9, 1-17.

\bibitem{}Eliaz, K. and Spiegler, R. (2011). ``Consideration Sets and Competitive Marketing.'' \textit{Review of Economic Studies} 78, 235-262.
 
\bibitem{}Falkinger, J. (2008) ``Limited Attention as a Scarce Resource in Information-Rich Economies.'' \textit{The Economic Journal} 118, 1596-1620.

\bibitem{}Grossman, G.M., Shapiro, C. (1984). ``Informative Advertising with Differentiated Products.'' \textit{Review of Economic Studies} 51, 63-81.

\bibitem{}Haan, M.A., Moraga-Gonzalez, J.L. (2011). ``Advertising for Attention in a Consumer Search Model.'' \textit{The Economic Journal} 121,552-579.

\bibitem{}Hefti, A., (2015).  ``Limited Attention, Competition and Welfare.''  Working Paper.  University of Zurich.

\bibitem{}Hefti, A., Liu, S. (2016).  ``Targeted Information and Limited Attention.''  Working Paper.  University of Zurich.   

\bibitem{}Hoyer, W. D. (1984). ``An Examination of Consumer Decision Making for a Common Repeat Purchase Product.'' \textit{Journal of Consumer Research} 11, 822-829.

\bibitem{}Janssen, M.C.W., and M.C. Non. (2005). ``Advertising and Consumer Search in a Duopoly Model.'' Tinbergen Institute Discussion Paper (TI 2005-022/1). Erasmus University, Rotterdam.

\bibitem{}Janssen, M.C.W., and M.C. Non. (2008). ``Advertising and Consumer Search in a Duopoly Model.'' \textit{International Journal of Industrial Organization} 26.1: 354-371. 

\bibitem{}Kaldor, N., (1950). ``The Economic Aspects of Advertising.'' \textit{The Review of Economic Studies} 18, 1-27.

\bibitem{}Lapersonne, E., Laurent, G., Le Goff, J. J. (1995). ``Consideration Sets of Size One: An empirical Investigation of Automobile Purchases.'' \textit{International Journal of Research in Marketing} 12(1), 55-66.

\bibitem{}Lee, M., Faber, R.J. (2007). ``Effects of Product Placement in On-line Games on Brand Memory: A Perspective of the Limited-Capacity Model of Attention.'' \textit{Journal of Advertising} 36(4), 75-90.

\bibitem{}Loginova, O. (2009). ``Exposure Order Effects and Advertising Competition.'' \textit{Journal of Economic Behavior and Organization} 71(2), pp. 528-538.

\bibitem{}Marshall, A. (1919). ``Industry and Trade: A Study of Industrial Technique and Business Organization; and of Their Influences on the Conditions of Various Classes and Nations.'' MacMillan and Co., London.

\bibitem{}Mitra, A. (1995).  ``Advertising and the Stability of Consideration Sets Over Multiple Purchase Occasions.''  \textit{International Journal of Research in Marketing} 12, 81-94.

\bibitem{}Narasimhan, C. (1988).  ``Competitive Promotional Strategies.''  \textit{Journal of Business} 61, 427-449.

\bibitem{}Robert, J., Stahl, D.O. (1993). ``Informative Price Advertising in a Sequential Search Model.'' \textit{Econometrica} 61, 657-686.

\bibitem{}Roberts, J. H., Lattin, J. M. (1997). ``Consideration: Review of Research and Prospects for Future Insights.'' \textit{Journal of Marketing Research} 34, 406-410.

\bibitem{}Shelegia, S., Wilson, C. (2016). ``A Generalized Model of Sales.'' \textit{ZBW: Leibniz Information Centre for Economics Working Paper}. 

\bibitem{}Stahl, D.O. (1989). ``Oligopolistic Pricing with Sequential Consumer Search.''  \textit{American Economic Review} 79, 700-712.

\bibitem{}Stahl, D.O. (1994). ``Oligopolistic Pricing and Advertising.'' \textit{Journal of Economic Theory} 64, 162-177. 

\bibitem{}Stahl, D.O. (2000).  ``Strategic Advertising and Pricing in E-Commerce'' in Baye, M.R. (ed.) \textit{Industrial Organization (Advances in Applied Microeconomics, Volume 9)} (Emerald Group Publishing Limited) 69-100.

\bibitem{}Varian, H.R., (1980).  ``A Model of Sales.''  \textit{American Economic Review} 70, 651-659.

\bibitem{}Williams, K. Petrosky, A., Hernandez, E. and Page Jr. R. (2011). ``Product Placement Effectiveness: Revisited and Renewed.'' \textit{Journal of Management and Marketing Research} 7, 1.

\end{thebibliography}


\end{document}